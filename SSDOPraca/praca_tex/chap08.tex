\chapter{Podsumowanie i wnioski}
\label{t:wnioski}

	\section{Porównanie implementowanych algorytmów}
	\label{t:wnioski:porownanie}
	
		\subsection{Wydajność i złożoność}
		\label{t:wnioski:porownanie:wydajnosc}
		
		% Szybkość działania
		% Skomplikowanie i użycie pamięci
		% tak
		
		Wydajność techniki SSDO-A w przedstawionej implementacji okazuje się być na akceptowalnym poziomie. Zgodnie z danymi w tabeli \ref{tab_7_A} jej użycie powoduje wzrost czasu wykonania jednego przebiegu programu o 0.91 ms. Przyjmując, że aby zachować płynność w przypadku komercyjnego silnika gry czas klatki nie powinien przekroczyć 16.67 ms (co odpowiada 60 FPS), jest to znacząca wartość, bo aż ok. 6\% tej liczby. Należy pamiętać, iż w zastosowaniach praktycznych podczas jednego przebiegu programu zachodzi mnóstwo dodatkowych obliczeń, m.in. tak czasochłonnych, jak fizyka, sztuczna inteligencja, czy inne efekty graficzne (np. dynamiczne cienie). Natomiast prezentowana w niniejszej pracy aplikacja testowa służy głównie do prostego renderingu siatek wielokątowych i nanoszenia na nie rezultatów działania omawianych technik okluzji kierunkowej.
		
		Wbrew wcześniejszym założeniom metoda SSDO-B okazała się dwukrotnie mniej wydajna od swojego pierwowzoru i aż prawie sześciokrotnie od SSDO-C. Na jej wykonanie potrzeba 2.41 ms, czyli 14\% czasu całkowitego. Jest to bardzo duża wartość i w konsekwencji dyskwalifikuje tę technikę z jakiegokolwiek wykorzystania praktycznego. Pomimo użycia dużo mniej wymagającego obliczeniowo procesu próbkowania okolicy piksela (zamiast kilkunastu czy kilkudziesięciu próbek na piksel ma miejsce tylko osiem samplowań) nie uzyskano spodziewanego przyrostu wydajności. Winowajcą jest technika warstw adaptacyjnych. Utworzenie SAT tylko dla głównych wersji buforów nie stanowi większego uszczerbku na wydajności (generator SAT uruchamiany jednokrotnie), lecz w przypadku metody adaptacyjnej sytuacja znacząco się zmienia. Obliczanie tablic sum ma miejsce aż pięciokrotnie. Oprócz tego, compute shader tworzący warstwy uruchomiony zostaje trzy razy. W rezultacie dochodzi do trzynastu wywołań funkcji \texttt{Dispatch} oraz sześciu zmian ustawionego programu cieniującego w jednym przebiegu renderingu. Ostatniej kwestii trudno uniknąć, ponieważ należy zachować odpowiednią kolejność operacji. Pomimo pracy na buforach o rozmiarach tylko 320x160 pikseli, wszystko to stanowi znaczący problem w kwestii szybkości działania. Oryginalna technika StatVO \cite{statvo} boryka się z omawianym problemem na mniejszą skalę, z~racji wykorzystania tylko bufora głębokości i opcjonalnie, wektorów normalnych. Wartości te można zmieścić na jednej teksturze. W przypadku SSDO-B używany jest też drugi bufor -- koloru. Podwaja to liczbę wykonywanych operacji. Zaleta omawianej implementacji to w~tym przypadku możliwość obliczania warstw adaptacyjnych oraz SAT jednocześnie dla obu tekstur, lecz wpływ na wydajność i tak jest niebagatelny. Pod kątem zapotrzebowania na pamięć SSDO-B przegrywa z pierwowzorem, lecz nieznacznie. Użyto większej liczby tekstur ale ich rozmiary są o połowę mniejsze.
		
		Sytuacja poprawia się bardzo w przypadku metody SSDO-C. Na wykresie \ref{wykr_7_A} widać wyraźnie, iż to najwydajniejsza z zaimplementowanych technik. Jej wpływ na przebieg programu jest niewielki, gdyż obliczenia zajmują tylko 0.12 ms, co stanowi zaledwie 0.7\% dopuszczalnego interwału. Szybkość przetwarzania wzrosła względem bazowej techniki o~ponad 2.5 raza. Uzyskano taki efekt poprzez usunięcie całego modułu SSDO-B odpowiadającego za SAT oraz warstwy adaptacyjne. Liczba obliczeń została znacząco zredukowana, gdyż tu wykonywane są tylko dwa przebiegi rozmycia gaussowskiego z zachowaniem krawędzi (poziome i pionowe). Także proces samplowania buforów jest uproszczony, ponieważ wystarczy pobrać jedną próbkę zamiast czterech dla każdej z warstw, nie trzeba też wyliczać wartości średnich ani różnic.
		W porównaniu do SSDO-A, zapotrzebowanie na pamięć jest niewiele większe. Jedyna różnica to wykorzystanie dwóch tekstur zawierających dane uśrednione oraz dwóch buforów tymczasowych. Wymiary wszystkich są jednak o połowę mniejsze niż wielkości tekstur wejściowych w SSDO-A.
		
		\subsection{Wygląd i występowanie artefaktów}
		\label{t:wnioski:porownanie:wyglad}
		
		% SSDO-A jest zaszumiona ale nie prezentuje artefaktów
		% SSDO-B ma problem z warstwami
		% SSDO-C wygląda w sumie najlepiej ale też ma problemy ze sobą
		% SSDO-B i -C nie tak dobrze oddają kierunkowość (patrz obszar badań)
		
		Obserwując zrzuty ekranu zamieszczone na rysunku \ref{fig_7_A} zauważyć można, że efekt generowany przez technikę SSDO-A jest lekko zaszumiony, co szczególnie widać we fragmencie D. Jako powód uznać można niedostateczną liczbę próbek lub niewystarczający stopień rozmycia. Prawdopodobnie efekt mógłby być lepszy, gdyby końcowy proces uśrednienia nastąpił kilkukrotnie. Pogorszyłoby to jednak i tak nie najlepszą wydajność algorytmu. Na obrazkach A i B widać, że tworzony efekt okluzji jest bardzo intensywny i dobrze zaznacza wszelkie kanty, zagięcia oraz inne ostre przejścia w renderowanej geometrii. Należy także zwrócić uwagę na niebagatelny wpływ kierunku oraz koloru światła, co szczególnie można zaobserwować na obrazku A. Kolor zacienienia po lewej stronie ma wyraźnie brązowy kolor, podczas gdy w SSAO mógłby być tylko odcieniem szarości. Na omawianym przykładzie widać także, że im bardziej geometria skierowana jest w stronę światła, tym mniej okluzji występuje w zagięciach schodów. Na obrazku C widnieje efekt działania oświetlenia pośredniego. U stóp sześcianów można zaobserwować jasną plamę o barwie odpowiadającej kolorowi pudełek. To oczekiwany i poprawny efekt, jednak pod kątem padania światła zbliżonym do 45 stopni na powierzchni ,,rzucającej'' promienie także się on pojawia (oznaczony czerwonym prostokątem), a jego źródło to podłoga. Czy można nazwać go pożądanym skutkiem działania metody, czy też raczej artefaktem wizualnym jest już kwestią subiektywną. Z~pewnością istnieje możliwość regulacji odpowiednim parametrem materiału. Na koniec należy zwrócić uwagę na fakt, że mimo wystąpienia szumu nie występują tu artefakty wizualne mające miejsce w następnych technikach.
		
		SSDO-B trapią największe problemy związane z błędami w wyświetlanym obrazie. Zostało to przedstawione na obrazkach A, C oraz D rysunku \ref{fig_7_B} i zaznaczone zielonymi okręgami. Widać, że w tych miejscach występuje ,,twarda'' granica między zacienieniem jasnym miejscem. Braki w okluzji wynikają z działania systemu warstw. Implementacja zakłada, że nieprzypisane miejsca wypełniane są zerami. Podczas pobierania uśrednionej wartości, gdy próbkowanie przebiega na granicy warstw, zera te mają wpływ na średnią. Doprowadza to do powstania braków w zacienieniu i jest kolejnym argumentem mogącym zdyskwalifikować ten algorytm z użycia praktycznego. W miejscach, gdzie okluzję widać, można zauważyć, że efekt znacznie różni się od SSDO-A. Z racji ewaluacji wolumetrycznego przesłonięcia poprzez wartość średnią szumy nie występują. Zacienienie przechodzi do jasnego koloru wyraźnym nieliniowym gradientem (obrazki A i D). Sam narożnik jest bardzo ciemny, jednak szybko okluzja zostaje wygaszona. Prowadzi to do powstania subtelniejszego rezultatu. Który z algorytmów prezentuje ładniejszy efekt to już kwestia subiektywna. Efekt światła pośredniego bywa gładszy i bardziej subtelny, nie występują także odbicia od podłogi.
		
		Wygląd okluzji generowanej przez SSDO-C pod względem nieistnienia błędów prezentuje się lepiej od SSDO-B, choć nie idealnie. Brak tu ,,dziur'' w zacienieniu, jest ono większości przypadków miękkie. Jednak ma miejsce inny rodzaj artefaktów. Na obrazkach B oraz D rysunku \ref{fig_7_C} widać wspomniany w rozdziale \ref{t:impl:c} defekt występujący na krawędziach (zaznaczone zielonym okręgiem). Zaproponowana metoda radzenia sobie z tym problemem nie jest w pełni skuteczna, lecz można zaobserwować jej działanie na obrazkach A i C. Drugi niepożądany błąd to ,,wylewanie się'' okluzji na płaskich powierzchniach leżących blisko krawędzi (obrazek D, czerwone prostokąty). Wynika on z faktu, iż prosta implementacja rozmycia z zachowaniem krawędzi, choć wydajna, powoduje zniekształcenie wartości w~okolicach miejsc nieciągłych. Warto nadmienić, iż kierunkowość zacienienia dużo trudniej zaobserwować niż w SSDO-A (obrazek A).
		Tak samo, jak w SSDO-B okluzja jest subtelna i otacza większe obszary niż w przypadku SSDO-A, co najlepiej widać po porównaniu obrazków B.
		
		Rysunek \ref{fig_7_D} dobitnie ukazuje, jak znaczący wpływ na wygląd generowanego obrazu ma okluzja otoczenia, niezależnie od przyjętej techniki oraz implementacji. Obrazek D~przedstawia niewiele więcej niż tylko płaską ścianę. Natomiast wszystkie pozostałe w lepszy lub gorszy sposób podkreślają kształty geometrii, wierniej symulują proces rozchodzenia się światła i w niebagatelnym stopniu przyczyniają się do polepszenia efektu wizualnego.
	
	\section{Zastosowanie i ograniczenia implementowanych algorytmów}
	\label{t:wnioski:zastosowanie}
	
	% brak próbkowania - dużo większa wydajność	
	% użycie buforów rozmytych o bardzo zredukowanym rozmiarze - większa wydajność
	% uproszczenie procesów fizycznych - większa wydajność
	% rozmycie buforów zamiast SAT może dodatkowo bardzo poprawić wydajność
	% StatVO z SAT nadaje się ŹLE bo za dużo danych, warstwy itp.
	
	% konieczność generowania SAT i warstw, chociaż mniejszych
	% mniej zgodny z fizyką efekt, bardziej bazujący na podobieństwie wyglądu
	% liczba warstw jest ograniczona i ma wpływ na wydajność
	% konieczność przetwarzania większej ilości danych niz w StatVO
	% braknie pamięci współdzielonej przy SAT przy większych buforach (będize trza podzielić na pojedyncze wywołania)
	% Technika subtrakcji kolorów może generować artefakty po użyciu tekstur.
	
	Z obu zaproponowanych nowych technik generowania kierunkowej okluzji otoczenia (SSDO-B i SSDO-C) pierwsza obwarowana jest największą liczbą ograniczeń. Po pierwsze, choć w założeniu miała być wydajniejsza od SSDO-A, jej wąskie gardło to proces generowania warstw adaptacyjnych oraz ich SAT. Należy także wspomnieć o tym, że to właśnie on jest przyczyną największej liczby restrykcji. Po pierwsze, jak wspomniano wcześniej, algorytm tworzenia SAT działający dla pojedynczego wiersza tekstury musi korzystać z pamięci współdzielonej, więc maksymalna szerokość buforów to 2048 pikseli. Rozwiązaniem byłoby przypisanie większej liczby obliczeń do jednego wątka lub podział tekstury na części i~tworzenie SAT dla każdej osobno, po czym łączenie. To jednak rodzi dodatkowe wywołania funkcji \texttt{Dispatch} i może znacząco obniżyć wydajność. Warto pamiętać także o limicie 4 warstw adaptacyjnych. W przypadku bardziej skomplikowanych scen zajdzie potrzeba stworzenia większej ich liczby, co pociągnie za sobą kolejne straty w szybkości działania i zwiększy użycie pamięci. Jednym z wniosków płynących z niniejszej pracy jest to, że algorytm SSDO-B źle się skaluje wraz ze wzrostem złożoności sceny i z powodu potrzeby przygotowania dużej ilości danych wejściowych nie uzyskuje wydajnościowej przewagi nad pierwowzorem -- SSDO-A. Zastosowanie w praktyce jest z tego powodu wątpliwe.
	
	W przypadku techniki SSDO-C rozwiązanie przedstawionego wcześniej problemu krawędzi sprawi, iż w pełni będzie nadawać się ona do użytku praktycznego jako prosta i~wydajna metoda kierunkowej okluzji otoczenia. Należy jednak zaobserwować, że generowanie oświetlenia pośredniego techniką różnicy kolorów (wzór \ref{t:algorytm:stat:indirect}) działa dobrze dla jednolitych obiektów, lecz w przypadku użycia tekstur może spowodować powstanie artefaktów graficznych. Pomimo tego mniej zgodny z fizyką, lecz wizualnie podobny do oryginału efekt wizualny spełnił postawione przed nim założenia. Występowanie niewielkich błędów jest problemem, lecz wydajnościowo SSDO-C dalece prześciga SSDO-A i dzięki temu pozostawia spore pole do rozwiązania tej niedogodności.
	
	Duża zaleta obu zaproponowanych algorytmów to brak konieczności uśredniania finalnego obrazu i całkowite wyeliminowanie problemu szumu lub kosztownego rozmycia, który trapi technikę SSDO-A.
	
	\section{Obszar przyszłych badań}
	\label{t:wnioski:przyszle}
	
	% Inne algorytmy rozmycia z zachowaniem krawędzi
	% Inne techniki naprawy krawędzi
	% Technika hybrydowa (trochę statvo trochę próbkowania)
	% Modulacja wielkości filtra rozmycia zależnie od kierunku światła (łatwiej w B trudniej w C); ewentualnie parametra wejściowego
	
	Aby móc zastosować w praktyce technikę SSDO-B, należy w pierwszej kolejności uprościć proces generowania SAT i warstw adaptacyjnych, a następnie rozwiązać problem z nieciągłością w okluzji pomiędzy warstwami. Pozwoli to na uzyskanie akceptowalnej wydajności i wyeliminowanie artefaktów. Rozwiązaniem pierwszej niedogodności mogłoby być np. skumulowanie wszystkich obliczeń w jeden globalny compute shader. Jednakże brak możliwości synchronizacji grup wątków sprawia, iż nie jest to trywialna operacja.
	
	Poprawiając technikę SSDO-C, warto byłoby skupić się na zastosowaniu bardziej złożonych algorytmów rozmycia z uwzględnieniem głębokości, które pozwolą uniknąć zniekształceń danych i ,,wylewania się'' okluzji.
	
	W kwestii naprawy krawędzi sensownym krokiem byłoby zastosowanie metody hybrydowej, tzn. dodatkowego próbkowania okolicy nieciągłości i na tej podstawie weryfikacja, czy może zajść w danym miejscu błąd, po czym naprawienie go. Dobrym pomysłem mogłoby także być użycie wydajnego algorytmu anty-aliasingu na buforze wejściowym oraz rozmytym, co zmniejszyłoby różnicę występującą na krawędziach.
	
	Jedną z wad statystycznej kierunkowej okluzji otoczenia jest mniejsze uzależnienie zacienienia od kierunku padającego światła. Różnicę tę najlepiej zaobserwować porównując obrazki A~rysunków \ref{fig_7_A} i \ref{fig_7_C}. Być może sensownym podejściem byłoby zastosowanie różnego rozmiaru filtra rozmycia lub różnej wielkości współczynników zacienienia w zależności od wartości kąta pomiędzy kierunkiem oświetlenia, a wektorem normalnym w danym miejscu.