\chapter{Podsumowanie i wnioski}
\label{t:wnioski}

	\section{Porównanie implementowanych algorytmów}
	\label{t:wnioski:porownanie}
	
		\subsection{Wydajność}
		\label{t:wnioski:porownanie:wydajnosc}
		
		\subsection{Wygląd i występowanie artefaktów}
		\label{t:wnioski:porownanie:wyglad}
	
	\section{Zastosowanie i ograniczenia implementowanych algorytmów}
	\label{t:wnioski:zastosowanie}
	
	% brak próbkowania - dużo większa wydajność	
	% użycie buforów rozmytych o bardzo zredukowanym rozmiarze - większa wydajność
	% uproszczenie procesów fizycznych - większa wydajność
	% rozmycie buforów zamiast SAT może dodatkowo bardzo poprawić wydajność
	
	% konieczność generowania SAT i warstw, chociaż mniejszych
	% mniej zgodny z fizyką efekt, bardziej bazujący na podobieństwie wyglądu
	% liczba warstw jest ograniczona i ma wpływ na wydajność
	% konieczność przetwarzania większej ilości danych niz w StatVO
	
	Fakt rezygnacji z próbkowania eliminuje lub znacznie ogranicza większość wymienionych w rozdziale \ref{t:algorytm:wady-ssdo} problemów. Zamiast kilkunastu czy kilkudziesięciu próbek na piksel ma miejsce tylko osiem samplowań, dysponujemy bowiem dwoma SAT -- bufora głębi i wektorów normalnych oraz bufora koloru.
	
	Prowadzi on do rezygnacji z próbkowania na rzecz modelu statystycznego, a tym samym zastąpienia przebiegów rozmycia przebiegami generowania SAT dla poszczególnych warstw.
	
	\section{Obszar przyszłych badań}
	\label{t:wnioski:przyszle}