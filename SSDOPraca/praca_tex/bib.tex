\begin{thebibliography}{999}

\bibitem{global-illum} \emph{Understanding Global Illumination} \url{ https://www.pluralsight.com/blog/film-games/understanding-global-illumination}, stan na dzień 12.09.2017

\bibitem{ao} S. Zhukov, A. Iones, G. Kronin, \emph{An Ambient Light Illumination Model}, Eurographics Rendering Workshop, 1998

\bibitem{statvo} Quintjin Hendrickx, Leonardo Scandolo, Martin Eisemann, Elmar Eisemann, \emph{Adaptively Layered Statistical Volumetric Obscurance}, High Performance Graphics, 2015

\bibitem{ssdo} Tobias Ritschel, Thorsten Grosch, Hans-Peter Seidel, \emph{Approximating Dynamic Global Illumination in Image Space}, Association for Computing Machinery, Inc., 2009

\bibitem{sloan} B. Loos, J. Sloan, \emph{Volumetric obscurance}, ACM SIGGRAPH, 2010

\bibitem{sat} Marcos Slomp, Toru Tamaki, Kazufumi Kaneda, emph {Screen-Space Ambient Occlusion Through Summed-Area Tables}, First International Conference on Networking and Computing, 2010

\bibitem{crytek} M. Mittring, \emph{Finding next gen: Cryengine 2}, ACM SIGGRAPH, 2007

\bibitem{luna} Frank D. Luna, \emph{Projektowanie gier 3D. Wprowadzenie do technologii DirectX 11}, Helion, 2014

\bibitem{prefix-sum} Mark Harris, \emph{Parallel Prefix Sum (Scan) with CUDA}, Nvidia, 2007

\bibitem{directx0} \emph{Windows Graphics Pipeline} \url{https://msdn.microsoft.com/en-us/library/windows/desktop/ff476882(v=vs.85).aspx}, stan na dzień 12.09.2017

\bibitem{directx1} \emph{What's new in Direct3D 11} \url{https://msdn.microsoft.com/en-us/library/windows/desktop/ff476346(v=vs.85).aspx}, stan na dzień 12.09.2017

\bibitem{directcompute} Eric Young, \emph{DirectCompute Optimizations and Best Practices}, Nvidia, GPU Technology Conference, 2010

\end{thebibliography}
