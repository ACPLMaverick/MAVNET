\documentclass[12pt, oneside, a4paper]{mwbk}
\usepackage[polish]{babel}
\usepackage[utf8]{inputenc}
\usepackage[OT4]{fontenc}

\usepackage{graphicx}
\usepackage{verbatim}

\usepackage[hidelinks]{hyperref}

\usepackage{titletoc}
\usepackage{etoolbox}

\usepackage{enumitem}
\usepackage[noend]{algorithmic}
\usepackage[english,ruled]{algorithm2e}

\usepackage{color}
\usepackage[normalem]{ulem}
\usepackage{listings}
\usepackage{rotating}
\usepackage{float}
\usepackage{textpos}

\usepackage{longtable}
\usepackage{colortbl}
\usepackage{booktabs}
\usepackage{pgfplotstable}
\usepackage{pdflscape}
\usepgfplotslibrary{units}
\usepackage{afterpage}

\newcommand\blankpage{%
	\null
	\thispagestyle{empty}%
	%\addtocounter{page}{-1}%
	\newpage}

\definecolor{dkgreen}{rgb}{0, 0.4, 0}


\linespread{1,3}
\oddsidemargin = 10pt
\textwidth = 470pt

\hyphenpenalty=1000
\tolerance=500

\setcounter{secnumdepth}{4}

\newcommand{\myfigure}[4]{
	\begin{figure}[H]
		\centering
		\scalebox{#3}
		{
			\includegraphics{#2}
		}
		\caption[#1]{#1}
		\label{#4}
	\end{figure}
}
\newcommand{\myownfigure}[4]{
	\begin{figure}[H]
		\centering
		\scalebox{#3}
		{
			\includegraphics{#2}
		}
		\caption[#1]{#1 Źródło: opracowanie własne.}
		\label{#4}
	\end{figure}
}
\newcommand{\myownextfigure}[4]{
	\begin{figure}[H]
		\centering
		\scalebox{#3}
		{
			\includegraphics{#2}
		}
		\caption[#1.]{#1 Źródło: opracowanie własne w oparciu o~źródło zewnętrzne\footnote{}.}
		\label{#4}
	\end{figure}
}
\newcommand{\myextfigure}[4]{
	\begin{figure}[H]
		\centering
		\scalebox{#3}
		{
			\includegraphics{#2}
		}
		\caption[#1.]{#1\footnotemark.}
		\label{#4}
	\end{figure}
}
\newcommand{\myextcitefigure}[5]{
	\begin{figure}[H]
		\centering
		\scalebox{#4}
		{
			\includegraphics{#3}
		}
		\caption[#1]{#1 #2.}
		\label{#5}
	\end{figure}
}


\pgfplotstableset{
	begin table=\begin{longtable}, % -------- CF
		end table=\end{longtable},
	every head row/.style={
		%before row=\caption{X-Powered-By header}\\\toprule, after row=\bottomrule \endhead,% --------- CF
		% as in the previous example, this patches the first row:
		before row={\hline},
		after row=\hline,
	},
	every last row/.style={% ------------ CF
		after row=\hline,
	},
	every even row/.style={
		before row={\rowcolor[gray]{0.92}}},
}

\pgfplotsset{compat=newest} 

\pgfplotsset{
	select row/.style={
		x filter/.code={\ifnum\coordindex=#1\else\def\pgfmathresult{}\fi}
	},
	discard if/.style 2 args={
		x filter/.code={
			\edef\tempa{\thisrow{#1}}
			\edef\tempb{#2}
			\ifx\tempa\tempb
			\def\pgfmathresult{inf}
			\fi
		}
	},
	discard if not/.style 2 args={
		x filter/.code={
			\edef\tempa{\thisrow{#1}}
			\edef\tempb{#2}
			\ifx\tempa\tempb
			\else
			\def\pgfmathresult{inf}
			\fi
		}
	}
}
	
\lstset{frame=tb,
	aboveskip=3mm,
	belowskip=3mm,
	showstringspaces=false,
	columns=flexible,
	basicstyle={\small\ttfamily},
	numbers=none,
	numberstyle=\tiny\color{gray},
	keywordstyle=\color{blue},
	commentstyle=\color{dkgreen},
	stringstyle=\color{mauve},
	breaklines=true,
	breakatwhitespace=true,
	tabsize=3}
	
\newcommand{\beginnumbered}{\begin{enumerate}[label*=\arabic*.]}

\lstdefinelanguage{HLSL}
{
	sensitive=true,
	morekeywords=[1]{
		attribute, const, uniform, varying,
		layout, centroid, flat, smooth,
		noperspective, break, continue, do,
		for, while, switch, case, default, if,
		else, in, out, inout, float, int, void,
		bool, true, false, invariant, discard,
		return, float2, float3, float4, float2x2, float2x3,
		float2x4, float3x2, float3x3, float3x4, float4x2,
		float4x3, float4x4, int2,
		int3, int4, bvec2, uint,
		uint2, uint3, uint4, lowp, mediump, highp,
		precision, SamplerState, sampler2D, sampler3D,
		samplerCube, sampler1DShadow,
		sampler2DShadow, samplerCubeShadow,
		sampler1DArray, sampler2DArray,
		sampler1DArrayShadow, sampler2DArrayShadow,
		isampler1D, isampler2D, isampler3D,
		isamplerCube, isampler1DArray,
		isampler2DArray, usampler1D, usampler2D,
		usampler3D, usamplerCube, usampler1DArray,
		usampler2DArray, sampler2DRect,
		sampler2DRectShadow, isampler2DRect,
		usampler2DRect, samplerBuffer,
		isamplerBuffer, usamplerBuffer, sampler2DMS,
		isampler2DMS, usampler2DMS,
		sampler2DMSArray, isampler2DMSArray,
		usampler2DMSArray, struct},
	morekeywords=[2]{
		radians,degrees,sin,cos,tan,asin,acos,atan,
		atan,sinh,cosh,tanh,asinh,acosh,atanh,pow,
		exp,log,exp2,log2,sqrt,inversesqrt,abs,sign,
		floor,trunc,round,roundEven,ceil,fract,mod,modf,
		min,max,clamp,mix,step,smoothstep,isnan,isinf,
		floatBitsToInt,floatBitsToUint,intBitsToFloat,
		uintBitsToFloat,length,distance,dot,cross,
		normalize,faceforward,reflect,refract,
		matrixCompMult,outerProduct,transpose,
		determinant,inverse,lessThan,lessThanEqual,
		greaterThan,greaterThanEqual,equal,notEqual,
		any,all,not,textureSize,texture,textureProj,
		textureLod,textureOffset,texelFetch,
		texelFetchOffset,TextureProjOffset,
		TextureLodOffset,TextureProjLod,
		TextureProjLodOffset,TextureGrad,
		TextureGradOffset,TextureProjGrad,
		TextureProjGradOffset,Texture1D,Texture1DProj,
		Texture1DProjLod,Texture2D,Texture2DProj,
		Texture2DLod,Texture2DProjLod,Texture3D,
		Texture3DProj,Texture3DLod,Texture3DProjLod,
		TextureCube,TextureCubeLod,shadow1D,shadow2D,
		shadow1DProj,shadow2DProj,shadow1DLod,
		shadow2DLod,shadow1DProjLod,shadow2DProjLod,
		dFdx,dFdy,fwidth,noise1,noise2,noise3,noise4,
		EmitVertex,EndPrimitive},
	morekeywords=[3]{
		gl_VertexID,gl_InstanceID,gl_Position,
		gl_PointSize,gl_ClipDistance,gl_PerVertex,
		gl_Layer,gl_ClipVertex,gl_FragCoord,
		gl_FrontFacing,gl_ClipDistance,gl_FragColor,
		gl_FragData,gl_MaxDrawBuffers,gl_FragDepth,
		gl_PointCoord,gl_PrimitiveID,
		gl_MaxVertexAttribs,gl_MaxVertexUniformComponents,
		gl_MaxVaryingFloats,gl_MaxVaryingComponents,
		gl_MaxVertexOutputComponents,
		gl_MaxGeometryInputComponents,
		gl_MaxGeometryOutputComponents,
		gl_MaxFragmentInputComponents,
		gl_MaxVertexTextureImageUnits,
		gl_MaxCombinedTextureImageUnits,
		gl_MaxTextureImageUnits,
		gl_MaxFragmentUniformComponents,
		gl_MaxDrawBuffers,gl_MaxClipDistances,
		gl_MaxGeometryTextureImageUnits,
		gl_MaxGeometryOutputVertices,
		gl_MaxGeometryOutputVertices,
		gl_MaxGeometryTotalOutputComponents,
		gl_MaxGeometryUniformComponents,
		gl_MaxGeometryVaryingComponents,gl_DepthRange},
	morecomment=[l]{//},
	morecomment=[s]{/*}{*/},
	morecomment=[l][keywordstyle4]{\#},
}

\renewcommand*{\lstlistlistingname}{Spis listing\'ow}

\begin{document}
\author{Marcin Wawrzonowski}
\title{Optymalizacja algorytmów generowania kierunkowej okluzji otoczenia\linebreak w przestrzeni ekranu}
\begin{titlepage}
\thispagestyle{empty}
\begin{textblock}{1}(-2.65,-1.65)
\includegraphics{figures/tytulowa_pusta_mgrinz.pdf}
\end{textblock}
\vspace{7.3cm}
\begin{center}
\fontfamily{ptm}
\selectfont
\Huge
Optymalizacja algorytmów generowania kierunkowej okluzji otoczenia\linebreak w przestrzeni ekranu
\end{center}
\begin{center}
\fontfamily{ptm}
\selectfont
Praca dyplomowa magisterska
\end{center}
\vspace{5.0cm}
\begin{center}
\fontfamily{ptm}
\selectfont
\hspace{-1cm}
\begin{tabular}{l}
Wydział Fizyki Technicznej, Informatyki i Matematyki Stosowanej \\
Promotor: dr inż. Dominik Szajerman \\
Dyplomant: inż. Marcin Wawrzonowski \\
Nr albumu: 207612 \\
Kierunek: Informatyka \\
Specjalność: Technologie Gier i Systemów Interaktywnych
\end{tabular}
\end{center}
\vspace{-.5cm}
\begin{center}
\fontfamily{ptm}
\selectfont
\begin{textblock}{13}(0,0.4)
Łódź, 2017
\end{textblock}
\end{center}
\end{titlepage}

\blankpage

\tableofcontents

\chapter{Wprowadzenie}
\label{t:wprowadzenie}

	% twórcy dążą do generowania realistycznej grafiki i wprowadzają oświetlenie globalne etc.
	% wymagający, niedokładny i podatny na błędy proces
	% ulepszenie wydajności kierunkowej okluzji
	
	Twórcy gier wideo i symulacji komputerowych od początku istnienia tych dziedzin dążyli do uzyskania obrazu jak najbardziej odpowiadającego rzeczywistości. Algorytmy obliczania oświetlenia wirtualnej sceny są prawdopodobnie najistotniejszymi elementami w~tym procesie. Ponieważ standardowy, oparty na śledzeniu promieni model światła globalnego był i ciągle jest zbyt kosztowny w aplikacjach czasu rzeczywistego, wielu twórców prześciga się w kreowaniu nowych metod przybliżania tej techniki. 
	
	Jedna z najpopularniejszych to SSAO -- okluzja otoczenia obliczana w przestrzeni ekranu, symulująca zacienienia w miejscach, do których w rzeczywistości docierałoby mniej światła. Powstało bardzo dużo algorytmów generujących SSAO, jednak dużą część z nich trapią problemy związane ze znaczącym wpływem na wydajność, artefaktami oraz szumami. Okluzja otoczenia w klasycznym ujęciu jest bowiem efektem o niskiej częstotliwości. Warto także wspomnieć, że nie wiąże się z w żaden sposób z procesem obliczenia oświetlenia, pomimo, iż generuje wiarygodny i przyjemny dla oka efekt.
	
	Jako ulepszenie omawianej techniki wprowadzono \emph{kierunkową} okluzję otoczenia (SSDO) \cite{ssdo}, której główną cechą jest połączenie z parametrami światła, takimi jak kierunek i kolor. W ramach tej metody obliczone zostaje także jedno odbicie światła pośredniego. Trapią ją jednak podobne bolączki, co SSAO. Oprócz tego stanowi dość kosztowny obliczeniowo proces. Głównym zadaniem niniejszej pracy jest poprawa wydajności SSDO przy jednoczesnym zachowaniu, lub nieznacznym pogorszeniu, walorów wizualnych. Dokonano tego przy pomocy statystycznego modelu wolumetrycznego przesłonięcia \cite{statvo} zamiast podejścia klasycznego.
		
	\section{Cel pracy}
	\label{t:wprowadzenie:cel}
	
	% wzięcie modelu SSDO
	% wzięcie StatVO i zaimplementowanie za jego pomocą	SSDO
	% finalnie dwie techniki różniące się procesem przygotowania danych
	% zbadanie wydajności wpływu wizualnego oraz porównanie
	% omówienie zastosowań, ograniczeń i możliwości rozwoju
	
	Istnieje wiele technik poprawy wydajności oraz ulepszenia efektów generowanych przez okluzję otoczenia. Jedną z nich jest wspomniana wcześniej metoda statystyczna (StatVO). Niniejsza praca stawia sobie za zadanie polepszenie wydajności SSDO przy wykorzystaniu algorytmów StatVO. Główną zmianą będzie rezygnacja z próbkowania okolicy przetwarzanego piksela na rzecz wcześniejszego obliczania wartości średniej wokół niego. Dzięki temu nowa technika powinna być wolna od szumów i pracować dużo szybciej od oryginału. Algorytmów obliczania oświetlenia SSDO nie da się bezpośrednio zastosować w metodzie statystycznej, zaproponowano więc nowe wzory, za pomocą których osiągnięto zbliżony efekt wizualny.
	
	Opracowano i zaimplementowano dwie techniki statystycznej kierunkowej okluzji otoczenia. Różnią się one procesem przygotowania danych, a wspomniane wyżej wzory pozostają takie same. Zaimplementowano także bazową metodę SSDO dla celów porównawczych. Zbadano wydajność każdej techniki w postaci osiąganej liczby klatek na sekundę. Wartości te porównano ze sobą, omówiono przyczyny, dla których jedna z metod pracuje szybciej, a~druga dużo wolniej. 
	
	Przedyskutowano przewagi obu nowych technik nad pierwowzorem oraz ich wady. Zaproponowano rozwiązania najbardziej kluczowych problemów. Wzięto pod uwagę także efekt wizualny i zbadano różnice pomiędzy efektem oraz intensywnością zacienienia, wpływ kierunku i koloru światła na te czynniki oraz występujące artefakty. Na koniec omówione zostały zastosowania technik, ich ograniczenia oraz możliwości rozwoju.
	
	
	\section{Założenia}
	\label{t:wprowadzenie:zalozenia}
	
	% krótki opis aplikacji
	% krótki opis platformy i narzędzi
	
	Na potrzeby pracy stworzono aplikację prezentacyjną pokazującą działanie wszystkich trzech zaimplementowanych technik kierunkowej okluzji otoczenia. Renderuje ona trójwymiarową scenę, której geometria zawiera dużą liczbę zagięć, kantów oraz wnęk, za sprawą czego uzyskany zostaje bogaty efekt okluzji. Obiekty oświetlone są światłem kierunkowym obracającym się wokół ustalonej osi. Dzięki temu można obserwować zmiany i~przejścia SSDO. Użytkownik kontroluje położenie oraz obrót wirtualnej kamery przy użyciu klawiatury. Może także przełączać wyświetlanie zaimplementowanych trzech technik SSDO lub wyłączyć je. Aplikacja w formie tekstowej oraz liczbowej prezentuje liczbę klatek na sekundę oraz czas rysowania jednej ramki, co pozwala szybko i łatwo ocenić wydajność aktualnie oglądanej metody SSDO.
	
	Stworzona aplikacja to mały silnik renderujący z możliwością definiowania obiektów w~scenie, wczytywania oraz renderowania zasobów, takich jak siatki wielokątowe lub napisy. Na gotową ramkę obrazu nałożona może zostać dowolna liczba post-procesów. Da się także zmieniać parametry kamery. Moduł graficzny oświetla obiekty korzystając z techniki renderingu odroczonego.
	
	Aplikacja została stworzona na platformę Windows przy użyciu technologii Direct3D 11, WINAPI oraz kilku pomniejszych bibliotek. Ważną kwestią jest użycie właśnie tej wersji API graficznego Microsoftu, gdyż udostępnia on shadery obliczeniowe, kluczowe w implementacji statystycznej okluzji otoczenia.
	
	Urządzenie testowe to średniej klasy komputer PC.
	
	\section{Zakres pracy}
	\label{t:wprowadzenie:zakres}
	
	% streszczenia rozdziałów
	
	Aby ułatwić Czytelnikowi zrozumienie wszystkich poruszanych zagadnień, rozdział \ref{t:teoria}~zawiera w sobie niezbędną wiedzę teoretyczną. Krok po kroku zostanie omówiona istota działania okluzji otoczenia. Dalej wyjaśniona będzie koncepcja SSAO, czyli okluzji otoczenia w przestrzeni ekranu. Następnie naświetlona zostanie idea statystycznego wolumetrycznego przesłonięcia, wraz z kluczowymi wzorami. Bardzo istotnym elementem jest tu proces generowania tablic sum (SAT), którego definicja także będzie przytoczona. Na koniec dogłębnie wyjaśniono zasadę działania algorytmu kierunkowej okluzji otoczenia oraz podano wzory, na bazie których obliczone zostaje oświetlenie bezpośrednie oraz pośrednie.
	
	Rozdział \ref{t:algorytm} skupia się na głównej idei niniejszej pracy, czyli połączeniu techniki SSDO z ideą statystycznego wolumetrycznego przesłonięcia (StatVO). Najpierw omówione zostają wady SSDO, a następnie opisano, w jaki sposób StatVO zostało do tej metody zastosowane. Podano kluczowe wzory dla obu przebiegów oświetlenia, omówiono różnice pomiędzy dwoma zaimplementowanymi technikami oraz poruszono kwestię użycia przez nie pamięci.
	
	Rozdział \ref{t:technologie} to opis zastosowanych w prezentowanej aplikacji technologii. Dla każdej z~nich zamieszczono krótki opis, a następnie wyjaśniono, jaką rolę pełni ona w programie oraz dlaczego to właśnie jej użyto.
	
	W rozdziale \ref{t:budowa} zaprezentowano architekturę aplikacji. Podano główne komponenty programu oraz przedstawiono połączenia pomiędzy nimi przy pomocy uproszczonych diagramów UML. Każdy z kluczowych komponentów został dokładnie omówiony, przy czym szczególną uwagę poświęcono modułowi renderingu odroczonego oraz post-procesów.
	
	Rozdział \ref{t:impl} skupia w sobie omówienie implementacji trzech występujących w niniejszej pracy technik generowania SSDO. W każdym przypadku pokazano, jak zostały zrealizowane założenia projektowe. Podano algorytmy działania oraz listę wszystkich czynności, które są wykonywane, by osiągnąć założony efekt. Najbardziej kluczowe elementy implementacji, w szczególności realizacje omówionych w rozdziałach \ref{t:teoria} i \ref{t:algorytm} wzorów, uzupełniono listingami kodu HLSL.
	
	W rozdziale \ref{t:wyniki} zaprezentowano wyniki działania każdego algorytmu. Wydajność przedstawiono w tabeli i na wykresie, a za jednostkę miary przyjęto liczbę klatek rysowanych przez program w ciągu sekundy. Następnie za pomocą zrzutów ekranu pokazano wizualny efekt działania każdej z technik.
	
	Ostatni rozdział \ref{t:wnioski} kryje w sobie przede wszystkim podsumowanie efektów działania zaimplementowanych technik. Uzyskane wyniki zostają omówione i porównane ze sobą. Szczególny nacisk położono na wady oraz ograniczenia omawianych metod. Przedyskutowano także ich zastosowania w praktyce, a także nakreślono kierunek dalszych możliwych badań w omawianym temacie.

\chapter{Algorytmy generowania okluzji otoczenia}
\label{t:teoria}

Okluzja otoczenia (\textit{ambient occlusion}) od początku ściśle związana jest z pojęciem oświetlenia globalnego. Proces ten uwzględnia, oprócz bezpośredniego wpływu promieni światła padających na daną powierzchnię, także ich dalszą drogę, jako promieni odbitych i rozproszonych. Pozwala to na symulację niebezpośredniej iluminacji sceny, efektu ,,wylewania się'' kolorów i osiągnięcia dużo bardziej zbliżonego do rzeczywistości wyglądu świata wirtualnego \cite{global-illum}. Niestety, do uzyskania poprawnych efektów wymagane jest wzięcie pod uwagę bardzo dużej liczby promieni lub fotonów, co prowadzi do olbrzymiego narzutu obliczeniowego i~w~klasycznym ujęciu (\textit{ray tracing}, \textit{path tracing}, \textit{radiosity}) wyklucza możliwość zastosowania tej techniki w aplikacjach czasu rzeczywistego.

Skutkiem takiego stanu rzeczy było powstanie wielu metod przybliżających efekty działania globalnego oświetlenia i cechujących się także akceptowalną wydajnością. Jedną z nich jest okluzja otoczenia. Celem tej techniki jest symulacja procesu oświetlenia sceny poprzez przyjęcie założenia, iż w miejsca, które otoczone są z wielu stron geometrią dociera mniej światła niż w obszary wyeksponowane, co doprowadza do powstania zaciemnień we wszystkich narożnikach, zagięciach, szczelinach i otworach \cite{ao}. Efekt możemy zobaczyć na poniższym rysunku \ref{fig_2_A} Poniżej zostaną omówione najważniejsze istniejące metody generowania okluzji oraz bazująca na nich kierunkowa okluzja otoczenia, będąca głównym zagadnieniem niniejszej pracy.

\myownfigure{Efekt działania okluzji otoczenia.}{figures/fig_2_A.jpg}{0.5}{fig_2_A}

	\section{Okluzja otoczenia bazująca na geometrii (AO)}
	\label{t:teoria:geom}
	
	Bazowy i najprostszy algorytm okluzji otoczenia polega na wykorzystaniu jako danych wejściowych lokalnej geometrii danego obiektu. Na jej podstawie zostaje obliczone, w jakim stopniu okolica danego punktu jest przysłonięta przez otaczającą ją siatkę modelu. Sedno działania algorytmu zostało zaprezentowane na rysunku \ref{fig_2_B}.
	
	\myextcitefigure{Próbkowanie geometrycznej okluzji otoczenia.}{\cite{statvo}}{figures/fig_2_B.png}{0.4}{fig_2_B}
	
	Dla każdego wierzchołka obiektu lub umiejscowionego na niej punktu, odzwierciedlającego położenie teksela przypisanej mu tekstury, generowana jest pewna liczba losowych promieni. W zależności od tego, czy wzięty został pod uwagę wektor normalny, czy też nie, promienie należy rozłożyć równomiernie po otaczającej półkuli lub sferze. Następnie obliczane są kolizje promieni z sąsiadującą geometrią, zazwyczaj przyjmując jakąś odległość graniczną, poza którą nie mają one znaczenia. Okluzję w tym punkcie określa stosunek liczby promieni, które weszły w kolizję z otoczeniem do liczby wszystkich promieni. AO została formalnie zdefiniowana w \cite{sloan} i jest dana następującym wzorem:
	
	\begin{equation}
	\mathit{AO}(\mathbf{x}, \vec{n}) = \frac{1}{\pi}\int_{\Omega}^{}\rho(\mathit{d}(\mathbf{x}, \vec{\omega},))\vec{n}\cdot \vec{\omega}d\vec{\omega}\ ,
	\end{equation}
	
	gdzie \textbf{x} oznacza pozycję w scenie, a \(\vec{n}\) wektor normalny w tym miejscu. \(\Omega\) reprezentuje kierunki próbkowania, a \textit{d} jest odległością do pierwszego przecięcia promienia z geometrią obiektu. Funkcja \(\rho\) to zazwyczaj pewna funkcja wygaszania (\textit{falloff}), która różnicuje wpływ danej próbki na finalną okluzję w zależności od jej odległości od punktu początkowego. Dzięki temu znaczenie ma dystans do obiektu zasłaniającego i można uzyskać pożądany efekt gładkich przejść od jasnego do ciemnego koloru.
	
	Przedstawiona powyżej prosta ewaluacja nawet niewielkiej liczby promieni dla każdego punktu geometrii jest i tak zbyt kosztowna, by móc zastosować ją w czasie rzeczywistym. W~praktyce często dla każdego modelu generowane są oddzielnie, poza potokiem renderingu, specjalne tekstury, zwane mapami okluzji otoczenia. Przykład znajduje się na rysunku \ref{fig_2_C} poniżej. Taka technika często bywa wystarczająco skuteczna, lecz ma istotne ograniczenia. Nie uwzględnia wpływu jednego obiektu na drugi oraz nadaje się do symulowania okluzji na obiektach animowanych tylko w ograniczonym stopniu. Często nie ma także zastosowania w przypadku dużych, modularnych elementów sceny, jak budynki, których koordynaty teksturowania zazwyczaj nie leżą w przedziale (0, 1).
	
	\myownfigure{Przykładowa mapa okluzji otoczenia.}{figures/fig_2_C.png}{0.2}{fig_2_C}
	
	\section{Okluzja otoczenia w przestrzeni ekranu (SSAO)}
	\label{t:teoria:ssao}
	
	Po raz pierwszy zastosowana w 2007 roku przez firmę Crytek na potrzeby gry Crysis \cite{crytek}. Pierwotna forma SSAO stanowi w praktyce przeniesienie opisanej powyżej metody AO z przestrzeni modelu do przestrzeni ekranu (G-Bufora) i jest aplikowana jako post-proces. Dane wejściowe to bufor głębokości i opcjonalny bufor wektorów normalnych. Dla każdego piksela w buforze obliczany jest stopień przesłonięcia, biorąc pod uwagę najbliższe otoczenie. Metoda próbkowania została przedstawiona na rysunku \ref{fig_2_D}.
	
	\myextcitefigure{Próbkowanie SSAO techniką punktową.}{\cite{statvo}}{figures/fig_2_D.png}{0.45}{fig_2_D}
	
	W pierwszej kolejności obliczane jest bezwzględne położenie kilkunastu losowych punktów, leżących w najbliższym sąsiedztwie punktu \textbf{x}. Następnie dla każdego z nich wyliczona zostaje pozycja w przestrzeni UV tekstury bufora. Samplując bufor głębokości w tych miejscach, otrzymuje się dla każdej próbki konkretną głębokość istniejącej tam geometrii. Można dzięki temu wyliczyć różnicę między nią a głębokością punktu próbkowania. Okluzja zostaje obliczona jako stosunek liczby punktów znajdujących się głębiej niż narysowana uprzednio geometria do liczby wszystkich punktów \cite{luna}. Podobnie jak w przypadku AO, warto zastosować w obliczeniach wektor normalny i ograniczyć rozkład punktów próbkowania do półkuli.
	
	\myextfigure{Błędy i szumy wynikajace z podpróbkowania SSAO}{figures/fig_2_E.jpg}{0.3}{fig_2_E}
	
	\footnotetext{\url{https://goo.gl/vDCeak}}
	
	Warto jednak wspomnieć, że próbkowanie i dokonywanie obliczeń na buforach o dużym rozmiarze także jest czasochłonne. Jako, że obliczana przy pomocy próbkowania okluzja jest efektem o niskiej częstotliwości i wymaga późniejszego rozmycia, nic nie stoi na przeszkodzie, by bufor okluzji występował w rozmiarach wynoszących połowę lub jedną czwartą rozmiarów tylnego bufora. Wiążą się z tym faktem liczne błędy, artefakty i, przede wszystkim, szum, który możemy zaobserwować na rysunku \ref{fig_2_E}. Część algorytmu odpowiedzialna za rozmycie i redukcję niedociągnięć często potrafi być dużo bardziej kosztowna niż obliczenie samej okluzji \cite{statvo}.
	
	\section{Wolumetryczne przesłonięcie i model statystyczny (StatVO)}
	\label{t:teoria:statvo}
	
	Omawiana w \cite{statvo} technika wolumetrycznego przesłonięcia (VO - \textit{Volumetric Obscurance}) podchodzi do teorii zagadnienia podobnie, jak metoda AO, tj. uzależnia stopień przesłonięcia od ilości geometrii znajdującej się w pewnym obszarze wokół punktu \textbf{x}. Jednak w~odróżnieniu od niej nie uzyskuje wyniku poprzez próbkowanie. Autorzy \cite{statvo} budują model statystyczny, oparty na \textit{średniej} wartości głębokości dookoła danego miejsca (\(\mu\) na rysunku \ref{fig_2_G}). Zakładają, że im większa różnica pomiędzy \(\mu\) a \textit{d}, tym więcej geometrii znajduje się w~przyjętym okolicznym obszarze i tym większe jest przesłonięcie. Zamiast sfery lub hemisfery zbudowany zostaje wokół punktu próbkowania skierowany w stronę kamery prostopadłościan.
	
	\myextcitefigure{Istota techniki VO.}{\cite{statvo}}{figures/fig_2_G.png}{0.45}{fig_2_G}
	
	Kluczową obserwacją jest tu fakt, iż można obliczyć \(\mu\) uśredniając cały bufor głębokości, czyli w efekcie rozmywając go. Zastosowana metoda rozmycia została opisana w podrozdziale \ref{t:teoria:statvo:sat}. Należy jednak poczynić założenie, iż głębia w obszarze całego bufora jest funkcją ciągłą, co w praktyce rzadko ma miejsce, i~może prowadzić do występowania błędów oraz przeszacowań przesłonięcia. Autorzy rozwiązują ten problem przy użyciu techniki omówionej w podrozdziale \ref{t:teoria:statvo:adaptive}.
	
	\myextcitefigure{Wykres funkcji StatVO.}{\cite{statvo}}{figures/fig_2_H.png}{0.45}{fig_2_H}
	
	Za obliczenie przesłonięcia odpowiada funkcja StatVO dana wzorem:
	
	\begin{equation}
	\mathit{StatVO}(\mathbf{x}) = \psi(\frac{z_{B}(\mathbf{x}) - \mu(\mathbf{x})}{z_{B}(\mathbf{x}) - z_{T}(\mathbf{x})})\ .
	\end{equation}
	
	\(z_{B}\) i \(z_{T}\) oznaczają odpowiednio tylną i przednią ściankę prostopadłościanu wyznaczającego obszar obliczeń. Funkcja \(\psi\) została skonstruowana tak, iż sprowadza do zera wszystkie wartości ujemne, dla przedziału (0, 1) zachowuje się liniowo, a powyżej argumentu o wartości 1 opada do zera z współczynnikiem kierunkowym zdefiniowanym przez użytkownika. Oddaje to efekt wypadania promieni poza obszar próbkowania w poprzednio omawianych technikach.
	
	Tak obliczona okluzja daje gładki, dokładny efekt zaciemnienia w przysłoniętych obszarach i nie wymaga dalszego rozmywania, co jest niewątpliwie dużą zaletą z punktu widzenia wydajności.
	
		\subsection{Tablica sum (SAT)}
		\label{t:teoria:statvo:sat}
		
		Kluczowym dla omawianej techniki aspektem jest szybkie, a zarazem wolne od błędów wygenerowanie uśrednionego bufora głębokości. Autorzy \cite{statvo} w swojej pracy wybrali algorytm generowania tablic sum (\emph{Summed Area Table}). SAT to taka tablica, w której każdy element jest sumą wszystkich elementów znajdujących się po lewej i powyżej niego. Doskonale obrazuje to rysunek \ref{fig_2_H}. Tablice sum wymagają użycia typu danych o wysokiej precyzji, ponieważ potrafią się w nich znaleźć bardzo duże liczby. Mają szerokie zastosowania, począwszy od rozpoznawania obiektów, a skończywszy na generowaniu głębi pola \cite{sat}.
		
		\myextcitefigure{Tablica sum.}{\cite{sat}}{figures/fig_2_I.png}{0.5}{fig_2_I}
		
		Formalnie wartość określonej komórki SAT dana jest wzorem:
		
		\begin{equation}
		\mathit{SAT}(x, y) = \sum_{x}^{i=1}\sum_{y}^{j=1}\mathit{Dane}(i, j)\ ,
		\end{equation}
		
		gdzie maksymalne wartości \(x\) i \(y\) to odpowiednio szerokość i wysokość wejściowej tablicy danych.
		
		Najistotniejszą cechą tablic sum jest fakt, iż znalezienie średniej wartości z dowolnego prostokątnego obszaru tablicy danych wymaga pobrania z SAT tylko czterech próbek, a~następnie podstawienia ich do wzoru:
		
		\begin{equation}
		\overline{x} = \frac{x_{A} - x_{B} - x_{C} + x_{D}}{P}\ ,
		\end{equation}
		
		gdzie \(x_{N}\) jest wartością pola SAT w narożniku uśrednianego obszaru, idąc zgodnie z~ruchem wskazówek zegara począwszy od prawego dolnego rogu, a \(P\) -- jego polem powierzchni. Oznacza to, że niezależnie od rozmiaru filtra, złożoność obliczeniowa rozmycia wynosi zawsze \(O(1)\) \cite{sat}.
		
		Dużo bardziej złożonym zagadnieniem jest algorytm generowania SAT. Proste, sekwencyjne podejście bazujące na wzorze (2.3) posiada złożoność obliczeniową \(O(n^2)\), co jest niepożądane w aplikacji czasu rzeczywistego. W niniejszej pracy SAT jest tworzone przy użyciu GPU, korzystając z techniki równoległych sum prefiksów opisanej w \cite{prefix-sum}. Algorytm pracuje na jednowymiarowej tablicy danych, jednak proces można bardzo łatwo wykonywać równolegle, uruchamiając go jednocześnie najpierw dla wszystkich wierszy, a następnie dla kolumn. Autor traktuje kolejne kroki swojej metody jak węzły drzewa i dzieli ją na dwie fazy. W pierwszej drzewo przechodzone jest od liści do korzenia, w każdym kroku dodane zostają do siebie wartości znajdujące się w sąsiednich węzłach, co prowadzi do powstania sum cząstkowych. Na koniec węzeł-korzeń przechowuje sumę wszystkich komórek. Proces ten obrazuje poniższy rysunek \ref{fig_2_J}.
		
		\myextcitefigure{Pierwsza faza algorytmu równoległych sum prefiksów.}{\cite{prefix-sum}}{figures/fig_2_J.png}{0.55}{fig_2_J}
		
		Druga faza polega na przejściu drzewa z powrotem, od korzenia do liści. Na miejsce głównego węzła zostaje wstawione zero, a następnie w każdym kroku algorytmu dany węzeł przekazuje swoją wartość lewemu dziecku. W prawym natomiast zostaje umieszczona suma wartości aktualnie przetwarzanego węzła i poprzednia wartość lewego dziecka.
		
		\myextcitefigure{Druga faza algorytmu równoległych sum prefiksów.}{\cite{prefix-sum}}{figures/fig_2_K.png}{0.55}{fig_2_K}
		
		Algorytm wykonuje w sumie \(O(n)\) operacji, więc jest wydajny dla dużych tablic danych, takich jak bufor głębokości. W rozdziale \ref{t:impl:b} omówione zostało także poprawne zarządzanie wątkami tak, by uniknąć konfliktów przy dostępie do pamięci współdzielonej. Dodatkowo podnosi to wydajność algorytmu.
		
		\subsection{Adaptacyjne warstwy głębokości}
		\label{t:teoria:statvo:adaptive}
		
		W głównej części rozdziału \ref{t:teoria:statvo} wspomniano o konieczności przyjęcia założenia ciągłości bufora głębi przy obliczaniu wartości \(\mu\). W praktyce rzadko kiedy występuje taka sytuacja. Odległości między obiektami mogą być znaczne, co owocuje skokowymi zmianami głębokości. W procesie generowania SAT zostaną one uśrednione. Jest to efekt niepożądany a jego skutkiem może być powstawanie zaciemnień w obszarach większych nieciągłości, gdzie nie powinno ich być. Autorzy \cite{statvo} próbują poradzić sobie z tym problemem stosując technikę generowania kilku różnych warstw bufora głębokości, których podział jest zależny od aktualnie rysowanej geometrii i skupia się wokół skokowych zmian głębi.
		
		\myextcitefigure{Istota techniki adaptacyjnych warstw głębokości.}{\cite{statvo}}{figures/fig_2_L.png}{0.3}{fig_2_L}
		
		Algorytm polega na początkowym obliczeniu SAT dla bazowego bufora głębokości. Następnie wartość każdego elementu zostaje porównana z wartością średnią. Jeżeli jest większa, trafia na podwarstwę A, w przeciwnym wypadku -- na B. Zamysł tej techniki opiera się na obserwacji, że wokół nieciągłości wszystkie piksele z fragmentu będącego bliżej ekranu trafią na jedną warstwę, gdyż ich wartości są wyższe od średniej. W efekcie granica warstw przebiega wzdłuż nieciągłości, co jest pożądanym wynikiem. Na koniec dla obu podwarstw algorytm oblicza SAT. Wykonywane są jego dwie rekurencje, co owocuje czterema warstwami głębokości. Kluczowe w późniejszym ich próbkowaniu jest utrzymywanie czterokanałowej tablicy indeksów, gdzie w każdym elemencie znajdują się wartości 0 lub 1, w zależności czy element należy do tej warstwy. Dla niej także trzeba wygenerować SAT lub zawrzeć indeksy w nieużywanych kanałach tekstur poszczególnych warstw.
		
		Algorytm próbkowania jest zgodny z następującym wzorem:
		
		\begin{equation}
		\mathit{StatVO}_{W}(\mathbf{x}) = \frac{1}{\sum_{i \in V} n_{i}(\mathbf{x})} \sum_{i \in V}\mathit{StatVO}_{i}(\mathbf{x})n_{i}(\mathbf{x})\ ,
		\end{equation} 
		
		gdzie \(\mathit{StatVO}_{i}\) to funkcja zdefiniowana we wzorze (2.2), obliczana dla warstwy o indeksie \(i\). \(n_{i}\), czyli liczba próbek na danej warstwie może zostać łatwo wyliczona dzięki pobraniu wartości z przetwarzanego obszaru SAT tablicy indeksów i obliczeniu sumy jedynek znajdujących się w nim. Dzięki funkcji \(\psi\) okluzja wynikająca z warstw o średniej głębokości dalekiej od głębokości \(\mathbf{x}\) nie będzie miała znaczenia w końcowym rezultacie.
	
	\section{Kierunkowa okluzja otoczenia (SSDO)}
	\label{t:teoria:ssdo}
	
	Technika ta jest naturalnym rozwinięciem SSAO, dążącym do jeszcze bardziej realistycznego symulowania globalnego oświetlenia. Głównymi nowościami są: uwzględnienie kierunku i koloru padającego światła przy generowaniu zaciemnień oraz wprowadzenie obliczeń jednego kroku iluminacji niebezpośredniej. Dzięki użyciu tych samych próbek do tworzenia dwóch różnych efektów oraz braku konieczności korzystania z dużej ilości dodatkowych danych, osiągnięto wydajność zbliżoną do standardowych technik SSAO.
	
		\subsection{Oświetlenie bezpośrednie}
		\label{t:teoria:ssdo:direct}
		
		SSDO, tak jak SSAO, opiera się na próbkowaniu, jednak Autorzy odchodzą od klasycznego podejścia, w którym światło i okluzja obliczane są w osobnych krokach renderingu, a~następnie łączone \cite{ssdo}. Zamiast tego wprowadza wzór:
		
		\begin{equation}
		L_{dir}(\mathbf{P}) = \sum_{i=1}^{N}\frac{\rho}{\pi}L_{in}(\omega_{i})V(\omega_{i})\cos\theta_{i}\Delta\omega\ .
		\end{equation}
		
		Dla każdego piksela mającego pozycję w scenie \(\mathbf{P}\) oświetlenie bezpośrednie obliczane jest z \(N\) kierunków próbkowania, równomiernie rozłożonych na półkuli. Każda z próbek oblicza iloczyn docierającego do niej światła \(L_{in}\), widoczności \(V\) i pewnej funkcji BRDF \(\frac{\rho}{\pi}\). Kluczową różnicą pomiędzy SSDO a SSAO jest inne traktowanie próbek, w zależności od tego, czy znajdują się pod geometrią, czy nad nią. W omawianej technice istnieje podział próbek na odpowiednio przesłaniające i widoczne. Oświetlenie bezpośrednie dla danego piksela obliczane jest tylko z kierunków od \(\mathbf{P}\) do próbek widocznych (niebieska strzałka i punkt C~na lewej części rysunku \ref{fig_2_M}). Oddaje to ideę mówiącą o tym, że im więcej geometrii wokół punktu, tym większe zaciemnienie, ale i bierze pod uwagę kierunek oraz kolor światła.
		
		\myextcitefigure{Istota techniki okluzji kierunkowej.}{\cite{ssdo}}{figures/fig_2_M.png}{0.55}{fig_2_M} 
		
		\subsection{Oświetlenie niebezpośrednie}
		\label{t:teoria:ssdo:indirect}
		
		Drugim krokiem techniki SSDO jest obliczenie jednego kroku światła niebezpośredniego, które mogłoby odbić się od jasno oświetlonych obiektów. Wykorzystane w tym celu zostały próbki określone jako przesłaniające oraz bufor światła bezpośredniego obliczony w~poprzednim przebiegu. Zgodnie z prawą częścią rysunku \ref{fig_2_M}, dla każdej z próbek pobierany jest odpowiadający jej oświetlony kolor oraz wektor normalny. Następnie iluminacja pośrednia zostaje wyliczona z poniższego wzoru:
		 
		\begin{equation}
		L_{ind}(\mathbf{P}) = \sum_{i=1}^{N}\frac{\rho}{\pi}L_{pixel}(1 - V(\omega_{i}))\frac{A_{s}\cos\theta_{s_{i}}\cos\theta_{r_{i}}}{d_{i}^2} \ ,
		\end{equation}
		
		gdzie \(d_{i}\) oznacza odległość punktu przysłaniającego \(i\) od źródła \(\mathbf{P}\), a \(\cos\theta_{s_{i}}\) i \(\cos\theta_{r_{i}}\) to w efekcie iloczyny skalarne odpowiednio między wektorami normalnymi wysyłającego oraz odbierającego piksela a kierunkiem padania światła (czarne przerywane strzałki na rysunku \ref{fig_2_M}). Wartość \(A_{s}\) służy do kontrolowania wielkości plamy iluminacji. Czynnik \((1 - V(\omega_{i})\) gwarantuje, że oświetlenie niebezpośrednie nie pojawi się tam, gdzie i tak występuje niedobór światła.


\chapter{Model statystyczny kierunkowej okluzji otoczenia}
\label{t:algorytm}


	\section{Wady bazowego algorytmu SSDO}
	\label{t:algorytm:wady-ssdo}
	
	% konieczność próbkowania i rozmycia 
	% użycie dwóch przebiegów renderingu
	
	Aby móc rozważać ulepszenie algorytmu SSDO, należy zastanowić się najpierw nad jego wadami. Jako, że niniejsza praca skoncentrowana jest wokół poprawienia wydajności, takie też kwestie zostaną rozpatrzone.
	
	Pierwsza i najważniejsza wadą SSDO to konieczność próbkowania otoczenia każdego piksela. Każda instrukcja samplowania tekstury w programie cieniującym stanowi istotne obniżenie wydajności. W przypadku omawianej techniki trzeba korzystać z pojedynczego bufora w każdym z dwóch przebiegów renderingu lub z dwóch buforów w jednym przebiegu (kolor oraz wektory normalne razem z głębokością). Mając, przykładowo, 14 próbek, generuje to w sumie 28 kosztownych wywołań instrukcji \texttt{Texture.Sample}\footnote{W języku HLSL.} na jeden piksel \cite{luna}. Redukując tę liczbę do znacznie mniejszej wartości da się bardzo przyspieszyć działanie programu.
	
	Kolejną kwestią związaną z próbkowaniem jest jego niska częstotliwość. Generuje to widoczny szum oraz artefakty, choć dzięki temu można korzystać z mipmap buforów i~przyspieszyć ich samplowanie. Mniejsze tekstury nie wpłyną znacząco na jakość obrazu. Niemniej jednak trzeba poświęcić dodatkowe przebiegi renderingu\footnote{Tzn. wywołania funkcji \texttt{Draw} lub \texttt{Dispatch} biblioteki graficznej.}, zazwyczaj co najmniej dwa, na stworzenie rozmytej wersji okluzji. Jest to kolejny proces, który wymaga pobierania wielu okolicznych próbek dla każdego piksela, ponadto należy brać pod uwagę głębokość każdej próbki tak, aby nie rozmyć okluzji w~miejscach nieciągłości i nie wywołać niepożądanego efektu ,,rozlewania się'' \cite{luna}. Wynika z~tego jasno, iż rozmycie jest dość kosztowym i niepożądanym z punktu widzenia wydajności procesem, który najlepiej ograniczyć do minimum albo w ogóle go uniknąć.
	
	Za kolejny niewygodny fakt można uznać to, że bazowa technika SSDO wymaga do działania dwóch przebiegów renderingu -- dla oświetlenia bezpośredniego i niebezpośredniego. Da się zauważyć w \cite{ssdo}, że ,,rozlewanie się'' kolorów występuje tylko w obszarach dobrze doświetlonych, a jego kolor jest bardzo zbliżony, lub wręcz taki sam, jak bazowy kolor materiału rzucających niebezpośrednie światło obiektów. Sensownym krokiem byłaby rezygnacja z buforu koloru oświetlonego jako wejścia do drugiego przebiegu techniki na rzecz buforu koloru \emph{nieoświetlonego}. Pozwoliłoby to połączyć kalkulację obu rodzajów iluminacji w jeden przebieg i zaoszczędzić na wywołaniu funkcji \texttt{Draw} przy niewielkim lub wręcz niezauważalnym pogorszeniu realizmu sceny.
	
	\section{Zastosowanie modelu statystycznego}
	\label{t:algorytm:stat}
	
	% cel - przyspieszenie przy akceptowalnym pogorszeniu jakości
	% użycie statystycznego zamiast próbkowania - z czym się wiąże
	% symulowanie próbek poprzez uzależnienie od kierunku i koloru (wzorek)
	% light bleeding symulowany poprzez rozmycie koloru i różnicę (wzorek)
	% rozmycie buforów w technice C.
	
	Głównym eksperymentem niniejszej pracy jest próba podniesienia wydajności algorytmu SSDO omówionego w rozdziale \ref{t:teoria:ssdo}, przy jednoczesnym akceptowalnym pogorszeniu wyglądu renderowanego obrazu, tj. zachowaniu tych samych efektów o, w ostateczności, nieznacznie gorszej jakości. Finalnie stworzono dwie autorskie techniki, oznaczone jako SSDO-B i~SSDO-C. Bazową metodę określa się jako SSDO-A.
	
		\subsection{Oświetlenie bezpośrednie}
		\label{t:algorytm:stat:direct}
		
		Kluczowym pomysłem stało się użycie opisanego w rozdziale \ref{t:teoria:statvo} statystycznego modelu wolumetrycznego przesłonięcia (StatVO). Rezygnując z próbkowania należy brać pod uwagę, że efekt SSDO trzeba osiągnąć zupełnie innym algorytmem, nie uwzględniającym wielu różnych kierunków oświetlenia bezpośredniego. W omawianym przypadku technika stała się dużo bardziej uproszczona, lecz generuje podobne efekty. Trzeba zauważyć, iż kolor oraz kierunek zacienienia SSDO jest ściśle zależny od koloru i kierunku padającego światła. Można wykorzystać ten fakt, żeby ograniczyć okluzję generowaną przez StatVO do miejsc słabiej doświetlonych. Dokonano tego przy pomocy iloczynu skalarnego średniego wektora normalnego oraz kierunku światła. Kolor zaciemnienia został uzależniony od koloru oświetlenia poprzez zwykłe przemnożenie pierwszego przez drugi. Ważną czynnością jest jednak uprzednia normalizacja drugiego wektora tak, aby jasność światła nie miała znaczącego wpływu na wynikową barwę zaciemnienia. Można proces ten opisać następującymi wzorami:
		
		\begin{equation}
		D = saturate(pow(1 - max(\vec{n_{avg}} \cdot \vec{L_{d}}, 0), p))\ ,
		\end{equation}
		
		\begin{equation}
		V = AD(1 - StatVO(\mathbf{x})) \frac{\vec{L_{c}}}{\overline{L_{c}}}\ .
		\end{equation}
		
		\(V\) jest końcowym kolorem oświetlenia bazującego na SSDO. \(A\) to odgórnie ustalony współczynnik jasności. Jako \(n_{avg}\) oznaczono średni wektor normalny w danym miejscu. Symbole \(L_{d}\) i \(L_{c}\) są odpowiednio kierunkiem i kolorem światła padającego na scenę.
		
		\subsection{Oświetlenie niebezpośrednie}
		\label{t:algorytm:stat:indirect}
		
		Proces obliczania oświetlenia niebezpośredniego także został znacząco uproszczony i~sprowadza się do odpowiedniej manipulacji buforem koloru nieoświetlonego. Okno filtra jego SAT jest większe niż w przypadku bufora normalnych-głębokości aby sprawić wrażenie bardziej rozległego efektu niż okluzja i wykroczyć poza jej obszar, tak jak wygląda to w~\cite{ssdo}. Zgodnie z sugestią w podrozdziale \ref{t:algorytm:wady-ssdo} algorytm wykonuje się w tym samym przebiegu renderingu. Działanie polega na obliczeniu różnicy koloru piksela oraz średniej wartości ,,wylanego'' koloru. Następnym krokiem jest przekształcenie otrzymanego wektora do przestrzeni HSV, po czym następuje odwrócenie komponentu barwy i ponowne przekształcenie na RGB. Uzyskana w ten sposób wartość zostaje dalej uzależniona od koloru padającego światła, średniego koloru oraz przeciwieństw współczynnika kierunkowego \(D\)~i~okluzji. Na koniec wynik dodawany jest do finalnego koloru danego piksela. Cały proces przedstawia się wzorem:
		
		\begin{equation}
		I = \rho(c_{\mathbf{x}} - c_{avg})c_{avg}\vec{L_{c}}(1 - D)(1 - StatVO(\mathbf{x}))\ ,
		\end{equation}
		
		gdzie \(\rho\) to funkcja przekształcająca do przestrzeni HSV, odwracająca komponent barwy, po czym transformująca z powrotem do RGB.
		
		\subsection{Różnice pomiędzy SSDO-B a SSDO-C}
		\label{t:algorytm:stat:diffs}
		
		Algorytm przedstawiony powyżej został użyty zarówno w metodzie SSDO-B, której implementację opisano w rozdziale \ref{t:impl:b}, jak i SSDO-C, omówionej w \ref{t:impl:c}. Warto jednak wspomnieć o tym, co różni obie te techniki. SSDO-B polega na bezpośrednim zastosowaniu algorytmów StatVO do tworzenia kierunkowej okluzji otoczenia. SSDO-C odchodzi w~pewnym stopniu od tego schematu, stosując inną, mniej skomplikowaną metodę generowania danych wejściowych. Zrezygnowano tu z użycia SAT, kierując się chęcią uproszczenia procesu przygotowania wartości średnich oraz zmniejszenia liczby przebiegów i użytych tekstur. Zastosowano uśrednienie buforów normalnych, głębokości i koloru przy użyciu rozmycia Gaussa z zachowaniem krawędzi. Takie podejście redukuje także liczbę próbek w finalnym procesie do jednej na każdy bufor. SSDO-C cechuje się znaczącym zwiększeniem wydajności kosztem wprowadzenia niepożądanych artefaktów wizualnych w~postaci widocznych czasem cienkich obrysów obiektów oraz ,,wylewania się'' okluzji na płaskich powierzchniach w~okolicach krawędzi.
		
		\pagebreak
		
		\subsection{Użycie pamięci}
		\label{t:algorytm:stat:memory}
		
		Znaczącym czynnikiem przy ocenie przydatności i wydajności algorytmów jest ich użycie pamięci. Jej ilość wykorzystywana przez technikę często rośnie wraz ze zwiększaniem szybkości działania. W tym przypadku, w większości sytuacji, przyrost jest na szczęście nieznaczny. 
		
		Bazowa metoda SSDO (SSDO-A) do działania potrzebuje dwóch dodatkowych buforów. Jednego do przechowywania podpróbkowanego, zaszumionego oświetlenia a drugiego do przeprowadzenia rozmycia metodą ping-pongową. Jako, że efekt ma niską częstotliwość, bufory te mogą być w mniejszym rozmiarze niż tylny bufor aplikacji. Należy także wspomnieć, że do wygenerowania równomiernie rozłożonych próbek utworzona zostaje specjalna tekstura losowych wektorów o rozmiarze 1024x1024 pikseli. 
		
		W przypadku SSDO-B sytuacja wygląda inaczej. Nie biorąc pod uwagę techniki adaptacyjnych warstw głębi, tak jak i w SSDO-A wykorzystuje się dwa dodatkowe bufory -- SAT dla wektorów normalnych i głębokości oraz dla koloru. W niniejszej pracy algorytm generowania SAT (omówiony w rozdziale \ref{t:impl:b}) wymaga użycia kolejnych dodatkowych dwóch tekstur-buforów. Nie można zastosować metody ping-pongowej -- SAT są obliczane dla obu tekstur naraz. Sprawa komplikuje się, gdy uwzględnimy adaptacyjne warstwy głębokości. Trzeba stworzyć przynajmniej kolejne cztery tekstury, a wartość ta będzie się znacząco zwiększać, gdy podjęta zostanie decyzja o wprowadzeniu większej liczby warstw ze względu na skomplikowanie sceny.
		
		Użycie pamięci techniki SSDO-C jest porównywalne z SSDO-B bez adaptacyjnych warstw. Podobnie jak tam, potrzeba tu dwóch tekstur na uśrednione wersje buforów oraz dwóch tekstur-buforów dla poprawnego przeprowadzenia procesu rozmycia. 
		
		Warto wspomnieć, że w przypadku SSDO-B i SSDO-C tekstury są rozmiarów jednej czwartej rozmiaru tylnego bufora aplikacji, dzięki czemu użycie pamięci nie odbiega znacząco od SSDO-A.
		

\chapter{Wykorzystane technologie}
\label{t:technologie}
	
	\section{Direct3D 11}
	\label{t:technologie:directx}
	
	Direct3D w wersji 11 jest częścią pakietu firmy Microsoft zwanego DirectX. Biblioteka ta to zaawansowane API graficzne, udostępniające zestaw funkcji do komunikacji z GPU i rysowania grafiki 2D oraz 3D. Udostępniony został programowalny potok renderingu, umożliwiający m.in. modyfikację geometrii (Geometry Shader) oraz teselację (Hull Shader, Domain Shader) \cite{directx0}.
	
	Powodem wykorzystania Direct3D w niniejszej pracy jest przede wszystkim mnogość i~elastyczność oferowanych przez niego możliwości, dopracowanie techniczne, pokaźna dokumentacja i dobra integracja z systemem operacyjnym Windows oraz najnowszymi kartami graficznymi. Pozwala to uzyskać dużą wydajność na potrzeby testów. Wieloplatformowość będąca domeną OpenGL nie ma tu znaczenia, ponieważ aplikacja w założeniu pracować ma tylko na systemie operacyjnym Windows i platformie PC.
	
	Kluczowym elementem Direct3D jest język HLSL, w którym tworzone są programy cieniujące, tzw. \emph{shadery}. Składnią i semantyką przypomina on C++, jednak jest od niego dużo prostszy. Zapewnia także mnóstwo wbudowanych, przydatnych funkcji matematycznych wykonujących obliczenia wspierane sprzętowo przez GPU. W niniejszej pracy HLSL ma szczególne znaczenie, ponieważ za jego pomocą zaimplementowane zostały wszystkie omawiane algorytmy renderingu, głównie w formie shaderów pikseli oraz obliczeniowych.
	
	Wersja 11 wprowadza kilka istotnych nowości i udogodnień, wśród nich można wymienić dynamiczne linkowanie shaderów, wielowątkowość czy shadery obliczeniowe \cite{directx1}. To właśnie te ostatnie są jednym z najistotniejszych narzędzi wykorzystanych w niniejszej pracy i zasługują na szczegółowe omówienie w następnym podrozdziale.
	
	\section{DirectCompute}
	\label{t:technologie:directcompute}
	
	% czym jest i na co pozwala
	% jak został wykorzystany (rw buffers)
	% przewaga nad innymi technologiami
	
	Jest to alternatywna nazwa dla shaderów obliczeniowych Direct3D 11\footnote{\url{https://blogs.msdn.microsoft.com/chuckw/2010/07/14/directcompute/}}. Technologia ta udostępnia możliwość wykonywania obliczeń GPGPU całkowicie lub częściowo w oderwaniu od potoku renderingu DirectX. Podobnie, jak w przypadku bliźniaczych API, takich jak CUDA czy OpenCL, programista otrzymuje dostęp do funkcji GPU niższego poziomu. Może on uruchomić określoną przez siebie liczbę wątków oraz ich grup i wykorzystać GPU do obliczeń niezwiązanych lub niebezpośrednio związanych z renderingiem.
	
	Olbrzymią zaletą DirectCompute jest całkowita integracja z API Direct3D 11, co umożliwia współdzielenie wszystkich zasobów np. pomiędzy pixel shaderem a compute shaderem. Zapewnia to znaczącą wygodę oraz wydajność. Użycie shadera DirectCompute wymaga po prostu ustawienia go w wirtualnym urządzeniu Direct3D, przypisanie mu zasobów wejściowych i wyjściowych, a następnie wywołania funkcji \texttt{Dispatch} uruchamiającej obliczenia.
	
	Wątki w compute shaderach podzielone są na wywołania (\textit{dispatch}) i grupy. Każde wywołanie to trójwymiarowa tablica grup, a jej rozmiary ustalane są przez parametry przesłane do wspomnianej wyżej funkcji uruchamiającej shader. Każda grupa jest trójwymiarową tablicą wątków, której wymiary zostają określone w kodzie HLSL i nie mogą być zmienione z poziomu aplikacji. Istotną cechą jest fakt, iż grupy posiadają wewnętrzną, szybką pamięć współdzieloną o ograniczonym rozmiarze. Została ona wykorzystana w~implementacji algorytmu generowania SAT. Maksymalny rozmiar grupy jest niewielki, dla Direct3D 11 i~technologii Pixel Shader 5 wynosi 1024 wątki. W obrębie grupy może także być wykonywana synchronizacja, działająca na zasadzie bariery. Wspólna pamięć ani koordynacja wątków nie istnieje pomiędzy grupami, jako, że ich liczba może być bardzo duża i nie ma gwarancji, iż wszystkie zostaną uruchomione równolegle \cite{directcompute}.
	
	Istotnym elementem compute shaderów wykorzystanym w niniejszej pracy są obiekty tekstur \texttt{RWTexture}, nadających się nie tylko do odczytu, ale i zapisu. Z tego względu zostały użyte do przechowywania SAT oraz rozmytych wersji buforów.
	
	\section{Biblioteki pomocnicze}
	\label{t:technologie:helpers}
	
	Poniżej przedstawiono listę pomocniczych bibliotek, które (za wyjątkiem WINAPI) nie są kluczowe do działania aplikacji, ale znacznie ułatwiły jej implementację i ulepszyły działanie.
	
		\subsection{WINAPI}
		\label{t:technologie:helpers:winapi}
		
		Zwane także Windows API, jest główną i najważniejszą biblioteką potrzebną do programowania aplikacji w systemie Windows. Udostępnia ona wszystkie niezbędne funkcje systemu operacyjnego. Zapewnia także kompatybilność z jego wcześniejszymi wersjami\footnote{\url{https://msdn.microsoft.com/en-us/library/windows/desktop/ff818516(v=vs.85).aspx}}.
		
		W przypadku omawianej aplikacji konieczne było skorzystanie z WINAPI w celu utworzenia okna programu i przekazanie jego uchwytu do Direct3D, wskazując mu miejsce, do którego ma renderować obraz.
		
		Niezbędnym elementem programu, stworzonym przy pomocy WINAPI, jest pętla komunikatów systemu operacyjnego. Dzięki niej aplikacja ,,wie'' m.in. kiedy należy narysować obraz, kiedy zostaje zamknięta lub kiedy kursor wkracza w obszar jej okna. Może także reagować na utratę kontekstu i na wszystkie istotne wydarzenia w systemie operacyjnym.
		
		\subsection{Raw Input}
		\label{t:technologie:helpers:rawinput}
		
		Raw Input jest elementem Windows API, jednak na tyle istotnym dla aplikacji, iż wspomniano o nim oddzielnie. Służy on do pobierania stanu urządzeń wejścia, takich jak klawiatura, mysz lub gamepad. Jednak w odróżnieniu od tradycyjnego modelu, w którym takie informacje aplikacja otrzymywała przy pomocy pętli komunikatów, Raw Input niesie ze sobą szereg udogodnień. Przede wszystkim, program dostaje dane tylko tych urządzeń, które zarejestruje, nie musi także otwierać ani zamykać w żaden sposób urządzenia. Jeśli jest ono nieaktywne lub odłączone, po prostu nie przyjdą od niego żadne komunikaty. Dane otrzymywane są w surowej formie, nieprzetworzonej przez system operacyjny, co zapewnia programiście dużą elastyczność działania\footnote{\url{https://msdn.microsoft.com/en-us/library/windows/desktop/ms645543(v=vs.85).aspx}}.
		
		W omawianej aplikacji Raw Input służy do zasilania informacjami prostego systemu obsługi kontrolerów użytkownika. Za jego pomocą realizowane jest głównie poruszanie oraz obracanie kamerą przy użyciu myszy i klawiatury, a także kilka pomniejszych funkcji omówionych bliżej w rozdziale \ref{t:budowa:inne:input}.
		
		\subsection{TinyObjLoader}
		\label{t:technologie:helpers:obj}
		
		Mała biblioteka służąca do ładowania plików OBJ zawierających siatki wielokątowe. Nie posiada żadnych zależności oprócz STL, co sprawia, że jest bardzo prosta w integracji z~własnym projektem. Znalazła zastosowanie w wielu popularnych technologiach, np. Bullet lub Cocos. Jest dostępna w Internecie na licencji FreeBSD\footnote{\url{https://syoyo.github.io/tinyobjloader/}}.
		
		Dzięki niej aplikacja jest w stanie sposób wczytać modele 3D, uniezależniając się od wyświetlania jedynie prostych brył. Warto w tym miejscu wspomnieć, że jednym z owych modeli jest katedra Sibenik autorstwa Marko Dabrovica\footnote{\url{http://hdri.cgtechniques.com/~sibenik2/download/}}.
		
		\subsection{FreeType}
		\label{t:technologie:helpers:freetype}
		
		Główną funkcją tej biblioteki jest wczytywanie plików czcionek w formacie wektorowym i rastrowym, a następnie renderowanie ich do tekstury. Wygenerowana w ten sposób tablica glifów służy następnie w omawianej aplikacji jako baza do rysowania napisów na ekranie. Dzięki niej można było m.in. pokazać aktualne statystyki renderingu, takie jak liczba klatek na sekundę. API cechuje mały rozmiar, wydajność i obsługa wielu popularnych platform. Jest dostępne na licencjach FreeType License oraz GNU GPL\footnote{\url{https://www.freetype.org/} oraz podstrony.}.
	
	\section{Visual Studio 2017}
	\label{t:technologie:vs}
	
	Jako środowisko programistyczne wykorzystano najnowszą wersję Microsoft Visual Studio. Wybór został dokonany zarówno ze względu na jego wszechstronność oraz możliwości, jak i~na dobrą znajomość. Warto wspomnieć o bardzo dobrej integracji IDE z pakietem DirectX. Pozwala ono m.in. na stworzenie gotowego projektu z zainicjalizowanym Direct3D (z czego nie skorzystano). Zawiera też w sobie obsługę formatu HLSL. Składa się na nią podświetlanie składni oraz, przede wszystkim, kompilator FXC. Dzięki niemu shadery są kompilowane oraz weryfikowane oddzielnie, przed uruchomieniem programu, do którego wczytany zostaje gotowy bajtkod.
	
	% czemu wybrano - wszechstronność i możliwości, łatwa integracja z API DirectX

\chapter{Budowa i działanie aplikacji}
\label{t:budowa}

	\section{Możliwości aplikacji}
	\label{t:budowa:mozliw}
	
	% co aplikacja robić może
	% rysować grafikę, oświetlenie phonga blinna, światło kierunkowe
	% rendering odroczony i system post-procesów
	% obsługa kamery i oglądanie sceny ze wszystkich stron
	% obliczanie oraz wyświetlanie statystyk i podpowiedzi
	
	Na potrzeby niniejszej pracy została stworzona aplikacja prezentacyjna. Jej głównym zadaniem jest wizualne zaprezen wanie działania omawianych technik kierunkowej okluzji otoczenia oraz pomiar ich wydajności. 
	
	W programie stworzona zostaje wirtualna scena, w której istnieje pewna geometria, charakteryzująca się znaczącą liczbą kantów, zagięć oraz szczelin. Symuluje to wygląd typowej sceny w komercyjnych aplikacjach czasu rzeczywistego i stanowi dobre pole do przetestowania technik okluzji otoczenia. Powinna ona wystąpić w dużej liczbie miejsc, ponadto nachodzące na siebie obiekty pozwalają sprawdzić algorytmy pod kątem podatności na występowanie artefaktów. W scenie umieszczono także kilka kolorowych sześcianów, dzięki czemu łatwo jest przebadać i zobaczyć efekt występowania oświetlenia niebezpośredniego.
	
	Wszystkie obiekty oświetlone zostały światłem kierunkowym o ściśle określonym kolorze. Odbicie zwierciadlane obliczane jest przy użyciu zwykłego równania Phonga-Blinna. Do modelu oświetlenia dodano także komponent \emph{ambient}, dzięki któremu nieoświetlone miejsca nie są kompletnie czarne i można zobaczyć efekt działania okluzji. Dodatkowo światło kierunkowe powoli obraca się wokół jednej z osi układu współrzędnych -- pozwala to na zaobserwowanie ruchu zacienienia i różnice między kierunkową a zwykłą okluzją otoczenia.
	
	Renderowana scena jest oprócz tego kompletnie statyczna, jednak przy użyciu myszy i~klawiatury użytkownik może swobodnie przemieszczać oraz obracać kamerę. Efekty działania algorytmów mogą być sprawdzone pod dowolnym kątem patrzenia, z dowolnego miejsca i odległości. Poza tym, wciskając odpowiedni klawisz, widz jest w stanie przełączać się pomiędzy efektami i dzięki temu może je porównywać dla tych samych ustawień kamery, bez potrzeby resetowania aplikacji.
	
	Program posiada jeszcze jedną istotną cechę. Dzięki implementacji modułu rysowania tekstu istnieje możliwość wyświetlania na ekranie informacji istotnych z punktu widzenia omawianych zagadnień. To takich zaliczają się m.in. czas potrzebny na narysowanie jednej pełnej ramki obrazu oraz liczba klatek renderowanych w okresie jednej sekundy. Ma to kluczowe znaczenie w kwestii badania wydajności, gdyż da się od ręki sprawdzić, która technika działa szybciej, a która wolniej. Wyświetlana jest także krótka informacja dla użytkownika, mówiąca mu o sterowaniu oraz czynnościach, jakie może wykonać w obrębie aplikacji.

	\section{Ogólna architektura aplikacji}
	\label{t:budowa:architektura}
	
	% UML Głównych komponentów
	% Opisy klas: System, Renderer, Scene
	% Mechanizm managementu zasobów
	
	Na rysunku \ref{fig_5_A} przedstawiono główne komponenty programu oraz zależności pomiędzy nimi. Przerywana strzałka biegnąca od klasy A do B oznacza, że klasa A korzysta z funkcji udostępnianych przez klasę B. Niepogrubiona nazwa klasy znaczy, iż jest ona singletonem. Przy oznaczeniu agregacji, jeżeli nie napisano inaczej, mowa o powiązaniu 1 -- 1.
	
	\myownfigure{Diagram przedstawiający główne moduły aplikacji.}{figures/fig_5_A.png}{0.4}{fig_5_A}
	
	Nadrzędną klasą programu jest klasa \texttt{System}. Jej główne zadanie to inicjalizacja wszystkich podsystemów oraz modułów, które wymagane są do pracy programu. Musi także stworzyć okno z odpowiednimi parametrami i przesłać jego uchwyt do singletona \texttt{Renderer}. Następnie utrzymuje główną pętlę aplikacji i wywołuje funkcje aktualizacji poszczególnych komponentów w odpowiedniej kolejności: najpierw obliczany jest nowy stan encji sceny, następnie zostaje narysowana ramka obrazu, po czym aktualizują się moduły \texttt{Timer} oraz \texttt{Input}. Klasa \texttt{System} odpowiada także za komunikację z systemem operacyjnym i odbieranie jego komunikatów przy użyciu metody WINAPI \texttt{PeekMessage}. Obsługiwane są takie wydarzenia jak zamknięcie aplikacji, ustawienie kursora (jest wyłączany) czy aktualizacja danych z API RawInput.
	
	Singleton \texttt{Renderer} odpowiada przede wszystkim za inicjalizację oraz poprawne skonfigurowanie urządzenia Direct3D. Jego główną rolą w cyklu działania aplikacji jest udostępnianie interfejsów Direct3D -- \texttt{ID3D11Device}, służącego do tworzenia i niszczenia zasobów oraz stanów renderingu, oraz \texttt{ID3D11DeviceContext}, przy pomocy którego aplikacja może ustawiać zasoby dla poszczególnych sekcji potoku renderingu oraz wywoływać funkcje rysujące i obliczeniowe. \texttt{Renderer} przechowuje także zestaw domyślnie skonfigurowanych stanów mieszania buforów i umożliwia szybkie przełączanie się między nimi, bez konieczności tworzenia ich podczas cyklu programu.
	
	Kluczowym element całego programu to klasa \texttt{Scene}. Przechowywane są tam wszystkie encje systemu, których logika jest obliczana i aktualizowana w pierwszym kroku pętli programu. W skład encji wchodzą m.in. \emph{obiekty} (klasa \texttt{Object}), będące po prostu statycznymi elementami sceny, złożonymi z siatki wielokątowej oraz materiału. Istotnym składnikiem klasy \texttt{Scene} są różnego rodzaju elementy oświetlenia. Aplikacja obsługuje światła \emph{ambient}, kierunkowe i punktowe, jednak te ostatnie nie mają zastosowania w~niniejszym projekcie, gdyż w implementacji nie są związane z efektem kierunkowej okluzji otoczenia. Kolejne składniki \texttt{Scene} to post-procesy (klasa abstrakcyjna \texttt{Postprocess} i klasy z niej dziedziczące) oraz teksty (klasa \texttt{Text}). Te dwa rodzaje elementów system rysuje już po wyrenderowaniu i~oświetleniu całej sceny. Najistotniejszym jej składnikiem jest jednak kamera (klasa \texttt{Camera}). To ona określa, z którego miejsca i z jakimi parametrami będą rysowane encje. Wraz z klasą \texttt{GBuffer}, zawiera w sobie także implementację renderingu odroczonego, dokładniej opisanego w podrozdziale \ref{t:budowa:rendering:rendering}.
	
	W omawianym projekcie \texttt{Scene} pełni także rolę menedżera zasobów. Zaliczają się do nich shadery (rozumiane jako komplety zestaw programów cieniujących oraz buforów stałych dla pojedynczego wywołania potoku renderingu), shadery obliczeniowe (rozróżniane z racji na inne sposoby wywołania oraz zastosowania), siatki wielokątowe, materiały i fonty. Każdy rodzaj zasobu znajduje się w odpowiadającym mu słowniku, gdzie rolę kluczy pełnią ciągi znaków -- ścieżki do plików na dysku. Kiedy występuje odwołanie do danego obiektu poprzez ścieżkę, najpierw zostaje sprawdzone, czy zasób o takiej ścieżce nie został już załadowany. Jeśli tak, zostaje zwrócony wskaźnik do niego. W przeciwnym wypadku następuje proces wczytania i umieszczeniu w słowniku. Przy wyłączeniu programu słowniki są czyszczone a~ich elementy usuwane. Dzięki temu inne moduły nie muszą przejmować się tym, czy zasób został wczytany, nie muszą także brać pod uwagę konieczności zwolnienia go.
	
	Ostatnim z istotnych składników sceny jest kontroler (klasa \texttt{Controller}). Korzystając z modułów \texttt{Timer} oraz \texttt{Input} zarządza on działaniem aplikacji. Odpowiada za logikę sterowania kamerą przez użytkownika oraz ustawianie i aktualizowanie wszystkich tekstów na ekranie. Za jego pomocą można także przełączać się między post-procesami, dba on o to, by jednocześnie aktywny był tylko jeden.
	
	\section{Działanie głównych modułów}
	\label{t:budowa:rendering}
			
		% UML zależności pomiędzy klasami Renderer, Camera, Scene i GBuffer
	
		Na rysunku \ref{fig_5_B} przedstawiono główne komponenty modułu renderingu oraz przepływ sterowania pomiędzy nimi.
		
		\myownfigure{Diagram przedstawiający elementy modułu renderingu.}{figures/fig_5_B.png}{0.4}{fig_5_B}
	
		\subsection{Rendering odroczony}
		\label{t:budowa:rendering:rendering}
		
		% Opisy mechanizmu działania Camera i GBuffer
		% Mieszanie oświetlenia z buforem
		% Pipeline G-Bufora
		% Pipeline shaderów
		
		Zastosowanie renderingu odroczonego (\emph{deferowanego}) w niniejszej aplikacji zostało podyktowane faktem, iż duża część danych uzyskiwana w tym procesie potrzebna jest do realizacji omawianych technik kierunkowej okluzji otoczenia, na przykład bufor wektorów normalnych i głębokości. Klasy \texttt{Camera} i \texttt{GBuffer} zawierają w sobie implementację renderingu deferowanego. Pierwsza z nich jako argument funkcji \texttt{Draw} otrzymuje wskaźnik do obiektu klasy \texttt{Scene}. Jest też jej klasą zaprzyjaźnioną, co pozwala na łatwy dostęp do wszystkich potrzebnych składników.
		
		Najistotniejszym elementem kamery jest jej G-Bufor. To zestaw buforów służących do obliczeń oświetlenia w procesie renderingu deferowanego, a w dalszej kolejności - także i~post-procesów. Na początku procesu rysowania jako cele renderingu zostają ustawione bufory koloru oraz wektorów normalnych i okluzji (funkcja \texttt{G-Buffera} \texttt{SetDrawMeshes}). Dwa ostatnie elementy mieszczą się w jednym buforze, jako iż razem stanowią wektor o czterech komponentach. Następnie wszystkie obiekty sceny są renderowane, wypełniając danymi ustawione bufory. Kolejny krok to narysowanie na nich oświetlenia. Dla każdego rodzaju światła zostaje ustawiony odpowiedni program cieniujący, po czym następuje rendering wszystkich świateł tego typu, tzn. obliczenie ich równań dla każdego piksela ekranu. Każde światło  Dane wejściowe to bufory będące w poprzednim kroku obiektami wyjściowymi. Na podobnej zasadzie zostaje wykonane rysowanie post-procesów oraz tekstów, opisane bliżej w~podrozdziałach \ref{t:budowa:rendering:postprocesy} i \ref{t:budowa:inne:fonty}.
		
		Należy zauważyć, że w takim podejściu po stronie shaderów zawartych w G-Buforze znajduje się obowiązek implementacji modelu oświetlenia. W przypadku renderingu wprost (\emph{forwardowego}) tę funkcję posiadają shadery znajdujące się w materiałach, a co za tym idzie -- przypisane do rysowanych obiektów.
		
		\subsection{Post-procesy}
		\label{t:budowa:rendering:postprocesy}
		
		% Opis klasy Postprocess i przykłady jej użycia w klasach potomnych
		% Shadery postprocesowe - generowanie vertexów
		
		Post-procesy to funkcje pracujące na wyrenderowanych uprzednio buforach G-Bufora i generujące określony efekt graficzny w przestrzeni ekranu. Algorytm obliczany jest na GPU, zazwyczaj osobno dla każdego piksela przy użyciu pixel shaderów lub compute shaderów. Za implementację tych technik w omawianej aplikacji odpowiada klasa abstrakcyjna \texttt{Postprocess}, definiująca interfejs niezbędny do komunikacji z G-Buforem.
		
		Obiekt klasy \texttt{Postprocess} posiada zawsze swój zestaw shaderów, wczytywanych przy inicjalizacji, oraz trzy istotne funkcje -- \texttt{SetPass}, \texttt{AfterPass} i \texttt{Update}. Dwie pierwsze są wywoływane w procesie rysowania, a trzecia poza nim, służy do wykonywania obliczeń niezwiązanych z renderingiem.
		
		Rozpoczynając rysowanie post-procesów, G-Bufor m.in. wyłącza ustawione poprzednio cele renderingu oraz ustawia na wejściu bazowe bufory, tzn. koloru, wektorów normalnych i głębokości. Zostają dla nich także wygenerowane mipmapy, gdyż jest to potrzebne do implementacji algorytmów SSDO-B i SSDO-C. Następnie dla każdego post-procesu wywołana jest jego funkcja \texttt{SetPass}, przygotowująca dane i wejścia shadera. Później ma miejsce rysowanie. Metoda \texttt{AfterPass} może być implementowana przez post-proces jeżeli wymaga on przeprowadzenia jakichś operacji po samym renderingu. Całość zostaje powtórzona tyle razy, ile post-proces deklaruje przebiegów. Przy ostatnim przebiegu zostają zamienione miejscami bufory wyjściowe tak, by finalny efekt rysowania trafił do bufora końcowego.
		
		G-Bufor zapewnia post-procesom zestaw dodatkowych buforów, przydatnych kiedy mają więcej niż jeden przebieg. Ich rozmiary są równe połowie rozmiarów tylnego bufora, co zapewnia niewielkie pogorszenie finalnej jakości obrazu przy znaczącym zwiększeniu wydajności.
		
		Warto wspomnieć o czynionej przy okazji procesu renderingu post-procesów niewielkiej optymalizacji. Funkcja \texttt{Draw} Direct3D zostaje wywołana bez przypisanego bufora wierzchołków, natomiast w argumencie podana jest liczba trzech. W shaderze wierzchołków następuje zastosowanie prostego algorytmu\footnote{\url{https://web.archive.org/web/20140719063725/http://www.altdev.co/2011/08/08/interesting-vertex-shader-trick/}}, który polega na wygenerowaniu pełnoekranowego trójkąta, wypełniającego całą przestrzeń obrazu. Zysk następuje w dwóch miejscach -- po pierwsze nie dowiązujemy bufora i nie trzeba pobierać z niego danych. Po drugie, rysując trójkąt zamiast czworokąta unikamy obniżenia wydajności związanego z rasteryzacją na krawędzi dwóch trójkątów biegnącej wzdłuż przekątnej ekranu. Zbędne fragmenty pełnoekranowego trójkąta zostają i tak szybko obcięte przez test nożyc.
	
	\section{Działanie pozostałych modułów}
	\label{t:budowa:inne}
	
		\subsection{Pomiary wydajności}
		\label{t:budowa:inne:profiling}
		
		% Opis działania profilera i mechanizmu pobierania czasu w Timerze
		
		Pomiary wydajności są ściśle związane z klasą \texttt{Timer} i mechanizmem pobierania aktualnego czasu systemu. W tym celu wykorzystane zostały funkcje \texttt{QueryPerformanceCounter} i \texttt{QueryPerformanceFrequency} biblioteki WINAPI. Pozwalają one pobrać odpowiednio aktualny czas systemu w cyklach zegara procesora oraz częstotliwość jego pracy. Obie wartości cechuje bardzo duża precyzja. Dzieląc jedną przez drugą otrzymuje się czas procesora w sekundach. Dzięki temu łatwo jest obliczyć czas potrzebny na wygenerowanie jednej ramki obrazu oraz liczbę rysowanych ramek w ciągu sekundy.
		
		Jako składnik klasy \texttt{Controller} istnieje także klasa \texttt{Profiler}. Odpowiada ona za gromadzenie obliczanych przez \texttt{Timer} wyników i wyświetlanie ich na ekranie w formie tekstowej. Służy także do ich uśredniania, aby zapobiec skokowym, szybkim zmianą wprowadzającym użytkownika w błąd i utrudniającym dokonanie pomiarów wydajności.
	
		\subsection{Obsługa kontrolerów}
		\label{t:budowa:inne:input}
		
		% Opis działania klasy Input i obsługi biblioteki RawInput
		
		
		
		\subsection{Rysowanie fontów}
		\label{t:budowa:inne:fonty}
		
		% Opis mechanizmu wczytywania fontów, generowania tekstur oraz meshy i rysowania ich na ekranie

\chapter{Wyniki testów symulatora}
\label{t:wyniki}

	\section{Czas wykonania}
	\label{t:wyniki:czas_wykonania}

		Czas wykonania jest rozumiany jako czas potrzebny na przetworzenie jednego pełnego kroku symulacji tkaniny. Wyrażony został w~milisekundach. To najważniejsze kryterium porównawcze, gdyż mówi o tym, jak bardzo obliczenia obciążają sprzęt, jak duży procent całości pracy silnika stanowią i~w efekcie -- czy działanie symulatora cechuje płynność. 
		
		Wpływ na czas wykonania ma ilość przetwarzanych danych, czyli gęstość siatki tkaniny, oraz wybrana implementacja. Zależności te przedstawiono w~formie tabel oraz wykresów, osobno dla każdej metody i~implementacji. Liczbę wierzchołków można w~aplikacji łatwo modyfikować, zmieniając liczbę krawędzi poziomych i~pionowych. Przyjęto zakres od siatki posiadającej \(10 \times 10 \) wszystkich krawędzi (100 wierzchołków) do \( 120 \times 120 \) (14400 wierzchołków), z~krokiem co 10 krawędzi poziomych i~pionowych.
		
		Warto wspomnieć, że do zachowania pełnej płynności obrazu na ekranie należy rysować jedną jego klatkę przynajmniej 30 razy na sekundę. Oznacza to, iż czas wykonania symulacji nie może być większy niż ok. 33 ms. Najbardziej satysfakcjonującym wynikiem byłoby osiągnięcie go niższego niż ok. 16 ms, co odpowiada 60 klatkom na sekundę -- to zazwyczaj maksymalna szybkość renderingu przy włączonej synchronizacji pionowej obrazu. Założono, że pozostałe obliczenia związane z~pracą silnika symulacji są pomijalnie krótkie.
		
		Przyjęto następujące oznaczenia:
		
		\begin{enumerate}
			\item C -- liczba wszystkich wierzchołków.
			\item MS-GPU-A -- Model masy na sprężynie, implementacja GPU, platforma Android.
			\item PB-GPU-A -- Model oparty na pozycji, implementacja GPU, platforma Android.
			\item MS-GPU-W -- Model masy na sprężynie, implementacja GPU, platforma Windows.
			\item PB-GPU-W -- Model oparty na pozycji, implementacja GPU, platforma Windows.
			\item MS-CPU-A -- Model masy na sprężynie, implementacja CPU, platforma Android.
			\item PB-CPU-A -- Model oparty na pozycji, implementacja CPU, platforma Android.
			\item MS-CPUx4-A -- Model masy na sprężynie, implementacja CPU (4 wątki robocze), platforma Android.
			\item PB-CPUx4-A -- Model oparty na pozycji, implementacja CPU (4 wątki robocze), platforma Android.
		\end{enumerate}
		
		\begin{figure}[H]
			\begin{tikzpicture}
				\begin{axis}[
				xlabel=C,
				ylabel=$t_{x}$,
				y SI prefix=milli,
				y unit=s,
				width=15cm,
				height=11cm,
				grid=major,
				legend style={at={(0.025, 0.76)}, anchor=west}
				]
				\addplot[orange, very thick] table [y=$t_{MS-GPU-A}$, x=C]{chart_6_1_a.dat};
				\addlegendentry{$t_{MS-GPU-A}$}
				\addplot[red, very thick] table [y=$t_{PB-GPU-A}$, x=C]{chart_6_1_a.dat};
				\addlegendentry{$t_{PB-GPU-A}$}
				\addplot[purple, very thick] table [y=$t_{PB-GPU-W}$, x=C]{chart_6_1_a.dat};
				\addlegendentry{$t_{PB-GPU-W}$}
				\addplot[yellow, dashed, thick] table [y=$t_{MS-GPU-W}$, x=C]{chart_6_1_a.dat};
				\addlegendentry{$t_{MS-GPU-W}$}
				\addplot[blue, very thick] table [y=$t_{MS-CPU-A}$, x=C]{chart_6_1_b.dat};
				\addlegendentry{$t_{MS-CPU-A}$}
				\addplot[cyan, very thick] table [y=$t_{PB-CPU-A}$, x=C]{chart_6_1_b.dat};
				\addlegendentry{$t_{PB-CPU-A}$}
				\addplot[green, very thick] table [y=$t_{MS-CPUx4-A}$, x=C]{chart_6_1_b.dat};
				\addlegendentry{$t_{MS-CPUx4-A}$}
				\addplot[olive, very thick] table [y=$t_{PB-CPUx4-A}$, x=C]{chart_6_1_b.dat};
				\addlegendentry{$t_{PB-CPUx4-A}$}
				\end{axis}
			\end{tikzpicture}
			\caption[Wykres zależności czasu wykonania od liczby wierzchołków.]{Wykres zależności czasu wykonania od liczby wierzchołków. Źródło: opracowanie własne.}
			\label{wykr_6_1}
		\end{figure}
		
				
				\begin{sidewaystable}
					\pgfplotstabletypeset
					[
					columns/$t_{MS-GPU-A}$/.style={fixed,fixed zerofill,precision=4},
					columns/$t_{PB-GPU-A}$/.style={fixed,fixed zerofill,precision=4},
					columns/$t_{MS-GPU-W}$/.style={fixed,fixed zerofill,precision=4},
					columns/$t_{PB-GPU-W}$/.style={fixed,fixed zerofill,precision=4},
					columns/$t_{MS-CPU-A}$/.style={fixed,fixed zerofill,precision=4},
					columns/$t_{PB-CPU-A}$/.style={fixed,fixed zerofill,precision=4},
					columns/$t_{MS-CPUx4-A}$/.style={fixed,fixed zerofill,precision=4},
					columns/$t_{PB-CPUx4-A}$/.style={fixed,fixed zerofill,precision=4},
					every head row/.append style={before row={%
							\caption{Czas wykonania dla różnych gęstości siatki.}%
							\label{tab_6_1}\\\toprule}
					}
					]
					{chart_6_1.dat}
				\end{sidewaystable}
		\newpage
		Wykres pokazuje dużą przewagę wydajnościową metod implementowanych na GPU. W~przypadku Androida, czas obliczeń jest niemalże stały, niezależnie od liczby wierzchołków tkaniny. Drobne wahania wynikają głównie z~błędu pomiaru (rzędu kilku ms). Niewielki wzrost czasu przetwarzania w~końcowej fazie testów może wynikać nie tyle z~samego narzutu obliczeniowego, ile z~rosnącej temperatury urządzenia i~związanego z tym stopniowego obniżania wydajności przez system operacyjny. 
		
		Niemożność uzyskania czasu obliczeń niższego niż ok. 12--15 ms wynika prawdopodobnie z~faktu wymuszenia synchronizacji pionowej przez implementację transformacyjnego sprzężenia zwrotnego w sterowniku karty graficznej Adreno. Jak można się było spodziewać, wersję GPU na platformie PC cechuje dużo większa wydajność. W~omawianym przypadku różnica jest niemal 300-krotna. Co ciekawe, problem z~synchronizacją pionową tu nie występuje, choć czas przetwarzania także utrzymuje się na stałym poziomie. 
		
		Osobną kwestią są implementacje na CPU. Można zauważyć, iż czas przetwarzania rośnie liniowo wraz z~liczbą wierzchołków i~bardzo szybko osiąga wartości, które uniemożliwiają generowanie płynnego obrazu. Jedynie dla niskiej gęstości siatki uzyskano przewagę nad GPU, z~racji wspomnianego wcześniej problemu. Widać także, że spadek wydajności dla implementacji z~użyciem 4 wątków roboczych jest ok. dwukrotnie mniejszy niż w~przypadku podejścia sekwencyjnego.
		
		W przypadku GPU nie zarejestrowano znaczących różnic w czasie wykonania pomiędzy metodami symulacji, aczkolwiek na CPU model oparty na pozycji osiągał dla dużych liczb wierzchołków minimalnie lepsze wyniki niż jego rywal.
		
		%\subsection{Model masy na sprężynie -- GPU -- Android}
		%\label{t:wyniki:czas_wykonania:ms_gpu_andro}
		
		
		%\subsection{Model oparty na pozycji -- GPU -- Android}
		%\label{t:wyniki:czas_wykonania:pb_gpu_andro}
		
		
	%	\subsection{Model masy na sprężynie -- GPU -- Windows}
	%	\label{t:wyniki:czas_wykonania:ms_gpu_pc}
		
		
	%	\subsection{Model oparty na pozycji -- GPU -- Windows}
	%	\label{t:wyniki:czas_wykonania:pb_gpu_pc}
		
		
	%	\subsection{Model masy na sprężynie -- CPU -- Android}
	%	\label{t:wyniki:czas_wykonania:ms_cpu_andro}
		
		
	%	\subsection{Model oparty na pozycji -- CPU -- Android}
	%	\label{t:wyniki:czas_wykonania:pb_cpu_andro}
		
		
	%	\subsection{Model masy na sprężynie -- CPU (4 wątki) -- Android}
	%	\label{t:wyniki:czas_wykonania:ms_cpux4_andro}
		
		
	%	\subsection{Model oparty na pozycji -- CPU (4 wątki) -- Android}
	%	\label{t:wyniki:czas_wykonania:pb_cpux4_andro}
		
	
	\section{Stabilność}
	\label{t:wyniki:stabilnosc}
	
		Drugim najważniejszym problemem symulacji jest jej niestabilność, rozumiana jako skłonność do wpadania siatki tkaniny w~niekontrolowane drgania, co w~efekcie może prowadzić do ,,eksplozji''. Nawet jeśli się tak nie stanie, ciągłe ruchy układu skutkują nierealistycznym efektem wizualnym. Zjawisko to jest więc bardzo niepożądane i~często zmusza do restartu symulatora.
		
		Trudno określić, które dokładnie parametry mają wpływ na stabilność tkaniny. Z~pewnością najważniejszym z~nich jest sztywność -- większe siły sprężystości bądź większy udział ograniczników mogą prowadzić do powstawania anomalii w~procesie symulacji. Dla modelu masy na sprężynie znaczenie w~redukcji drgań ma także współczynnik ich tłumienia. Nie bez wpływu pozostają też takie zmienne, jak gęstość siatki, masa czy siła grawitacji.
		
		Na potrzeby testów wybrano jeden z~położonych w środku tkaniny wierzchołków oraz zbadano jego drgania w stanie spoczynku, tj. średnią różnicę pomiędzy położeniem obecnym a~poprzednim, w~każdym kroku symulacji. Pomiarów dokonano dla różnych współczynników sztywności, a następnie przedstawiono tę zależność w~postaci tabel i~wykresu. Przy każdej metodzie zostały zbadane dwa przypadki, uwzględniające inne masy, siły grawitacji, współczynniki tłumienia oraz gęstości siatki. Stan spoczynku określono jako stan, w którym tkanina opadła swobodnie z~pozycji poziomej do pionowej, zawieszonej w~dwóch punktach, i~przestała się poruszać. Warto przypomnieć, że dla modelu opartego na pozycji parametr sztywności (\(s\)) został odpowiednio przeskalowany tak, by mieścił się w wymaganym zakresie [0, 1] i niósł ze sobą podobny efekt, co jego odpowiednik w~modelu masy na sprężynie. Platformą testową jest mobilna wersja aplikacji, z~implementacją na GPU.
		%\newline
		
		Pomiar pierwszy -- sztywność: [50, 600], krok 50; masa: 0.2 \(kg\); grawitacja: 1 \(\frac{m}{s^2}\); współczynnik tłumienia: -0.5; gęstość siatki: 625 wierzchołków.
		
		Pomiar drugi -- sztywność: [50, 600], krok 50; masa: 0.7 \(kg\); grawitacja: 2 \(\frac{m}{s^2}\); współczynnik tłumienia: -10; gęstość siatki: 6400 wierzchołków.
		%\newline
		
		Drgania (\(d_{x}\)) podano w mikrometrach. Kolejne testy oznaczono zgodnie ze wzorem: \(A-n\), gdzie \(A\) -- rodzaj zastosowanego modelu symulacji (MS -- masy na sprężynie, PB -- oparty na pozycji), \(n\) -- numer pomiaru.
		\newline
		
		\pgfplotstabletypeset
		[
		every head row/.append style={before row={%
				\caption{Drgania w zależności od współczynnika sztywności.}%
				\label{tab_6_2}\\\toprule}}
		]
		{chart_6_2.dat}
		
		\begin{figure}[H]
			\begin{tikzpicture}
				\begin{axis}[
				xlabel=s,
				ylabel=$d_{x}$,
				y SI prefix=micro,
				y unit=m,
				width=14cm,
				height=10cm,
				grid=major,
				legend style={at={(0.025, 0.84)}, anchor=west},
				every axis y label/.style={
					at={(-0.12, 0.5)}, rotate=90,
					anchor=east}
				]
				\addplot[orange, very thick] table [y=$d_{MS-01}$, x=s]{chart_6_2.dat};
				\addlegendentry{$d_{MS-01}$}
				\addplot[green, very thick] table [y=$d_{PB-01}$, x=s]{chart_6_2.dat};
				\addlegendentry{$d_{PB-01}$}
				\addplot[red, very thick] table [y=$d_{MS-02}$, x=s]{chart_6_2.dat};
				\addlegendentry{$d_{MS-02}$}
				\addplot[olive, very thick] table [y=$d_{PB-02}$, x=s]{chart_6_2.dat};
				\addlegendentry{$d_{PB-02}$}
				\end{axis}
			\end{tikzpicture}
		\caption[Wykres zależności drgań od współczynnika sztywności.]{Wykres zależności drgań od współczynnika sztywności. Źródło: opracowanie własne.}
		\label{wykr_6_2}
		\end{figure}

		\myownfigure{Niestabilności występujące w obu modelach symulacji.}{figures/pic_6_1.png}{0.38}{pic_6_1}
		\newpage
		
		Główna różnica pomiędzy modelami masy na sprężynie i~opartym na pozycji ukazuje się właśnie tutaj. Widać, że w~pierwszym przypadku, dla pierwszej próby, z~początku zarejestrowano najniższą ze wszystkich oscylację drgań, jednak rośnie ona szybko wraz ze wzrostem parametru sztywności, dla najwyższej jego wartości doprowadzając nawet do ,,wybuchu'' symulacji. Jeśli chodzi o~drugie podejście, można zaobserwować duże oscylacje praktycznie niezależnie od elastyczności tkaniny, co pozwala wnioskować, iż zagęszczanie siatki także ma niebagatelny wpływ na drgania. Były one obecne praktycznie przez cały czas symulacji, widać je na rysunku \ref{pic_6_1} (po lewej). Ciągle poruszające się drobne zniekształcenia bardzo negatywnie wpływają na odbiór wizualny i~w~jakichkolwiek zastosowaniach praktycznych byłyby nie do zaakceptowania.
		
		Testy udowodniły, iż model oparty na pozycji cechuje wyjątkowa stabilność -- oscylacje są czasem nieznacznie większe niż u~rywala, jednakże w~obu próbach utrzymywały się na stałym poziomie, niezależnie od zwiększania parametru sztywności czy liczby wierzchołków. Drugi test ukazał jednak, że dla małej elastyczności i~gęstej siatki, tkanina zaczyna wchodzić w~niekontrolowane kolizje z samą sobą. Jest ona na tyle sztywna, by przy odpowiednim ułożeniu cząstek masy doprowadzić do ,,zawiśnięcia samej na sobie'' i~unieruchomieniu się w~powietrzu, de facto ignorując siłę grawitacji. Efekt ten można zaobserwować w~prawej części rysunku \ref{pic_6_1}. Takie zachowanie tkaniny także jest nie do zaakceptowania w warunkach praktycznych, jednak należy zaznaczyć, iż nie dochodzi tu do ,,wybuchu'', a~niestabilności nie mają charakteru drobnych, szybkich drgań, a raczej niekontrolowanego powolnego falowania. Spadek mierzonych oscylacji w~przypadku współczynnika sztywności większego niż 450 w~drugiej próbie da się wytłumaczyć sytuacją widoczną na rysunku \ref{pic_6_1} (po prawej). Silne falowanie miało miejsce głównie w~części siatki oznaczonej czerwonym prostokątem, a~obszar środkowy, z~którego pobrana została próbka, pozostawał we względnym spoczynku.
		\newpage
	%	\subsection{Model masy na sprężynie}
	%	\label{t:wyniki:stabilnosc:ms}
		
		
	%	\subsection{Model oparty na pozycji}
	%	\label{t:wyniki:stabilnosc:pb}
		
		
	\section{Efekt wizualny}
	\label{t:wyniki:efektwiz}
		
		Ostatnie kryterium oceny to po prostu stopień, w jakim zachowanie i~wygląd symulowanej tkaniny odzwierciedla rzeczywistość. Wyznacznik ten jest całkowicie subiektywny, jednak na pewno można zauważyć wprost proporcjonalną zależność pomiędzy jakością a~gęstością siatki. Mała liczba wierzchołków fizycznie nie pozwala na wygenerowanie realistycznych zmarszczek ani zagięć, tak charakterystycznych elementów animacji tkanin. Dla każdego modelu symulacji zaprezentowane zostaną zrzuty ekranu pokazujące zależność ,,efektu wizualnego'' od różnych parametrów, a~w~szczególności -- gęstości siatki. Platformą testową jest mobilna wersja aplikacji.
		
		Na zrzutach ekranu wchodzących w~skład rysunku \ref{pic_6_2} przedstawiono wygląd tkaniny symulowanej modelem masy na sprężynie. Zgodnie z~ruchem wskazówek zegara, począwszy od lewego górnego obrazka przyjęto następujące parametry:
		
		\begin{enumerate}
			\item Siatka \(10 \times 10\) krawędzi, współczynnik sztywności 200, współczynnik tłumienia -0.5, przyspieszenie grawitacyjne 5 \( \frac{m}{s^2} \), masa 0.8 \(kg\);
			\item Siatka \(40 \times 40\) krawędzi, współczynnik sztywności 500, współczynnik tłumienia -3.3, przyspieszenie grawitacyjne 1 \( \frac{m}{s^2} \), masa 0.9 \(kg\);
			\item Siatka \(80 \times 80\) krawędzi, współczynnik sztywności 500, współczynnik tłumienia -3.3, przyspieszenie grawitacyjne 0.5 \( \frac{m}{s^2} \), masa 0.5 \(kg\);
			\item Siatka \(120 \times 120\) krawędzi, współczynnik sztywności 700, współczynnik tłumienia -10, przyspieszenie grawitacyjne 0.5 \( \frac{m}{s^2} \), masa 0.5 \(kg\).
		\end{enumerate}
		
		Natomiast na rysunku \ref{pic_6_3} pokazano efekt wizualny dla modelu opartego na pozycji:
		
		\begin{enumerate}
			\item Siatka \(10 \times 10\) krawędzi, współczynnik sztywności 50, przyspieszenie grawitacyjne 5 \( \frac{m}{s^2} \), masa 2 \(kg\);
			\item Siatka \(40 \times 40\) krawędzi, współczynnik sztywności 100, przyspieszenie grawitacyjne 5 \( \frac{m}{s^2} \), masa 0.1 \(kg\);
			\item Siatka \(80 \times 80\) krawędzi, współczynnik sztywności 200, przyspieszenie grawitacyjne 5 \( \frac{m}{s^2} \), masa 0.1 \(kg\);
			\item Siatka \(120 \times 120\) krawędzi, współczynnik sztywności 300, przyspieszenie grawitacyjne 0.5 \( \frac{m}{s^2} \), masa 0.1 \(kg\).
		\end{enumerate}
		\newpage
		% kolizje z boksem
			
		\myownfigure{Wygląd tkaniny dla różnych parametrów (model masy na sprężynie).}{figures/pic_6_2.png}{0.38}{pic_6_2}
		
		\myownfigure{Wygląd tkaniny dla różnych parametrów (model oparty na pozycji).}{figures/pic_6_3.png}{0.38}{pic_6_3}
		\pagebreak
		
		% podobieństwa między modelami, różnice między modelami, co się dzieje dla małej ilości krawędzi, co się dzieje dla dużej ilości krawędzi, kolizje z boksem, kolizje z samym sobą
		
		Jak może się wydawać z~pobieżnych oględzin zrzutów ekranu, rozbieżności w~wyglądzie tkaniny symulowanej różnymi modelami nie są duże. Faktem jest, iż dla małej liczby krawędzi tkanina wygląda i~zachowuje się niemal tak samo. Zmarszczki i~kształt ułożenia na obiekcie wyglądają tym lepiej i~bardziej realistycznie, im gęstsza jest siatka, zarówno w pierwszej, jak i~w~drugiej metodzie. 
		
		Podobieństwa kończą się jednak w momencie porównania parametrów, jakich użyto do uzyskania podobnych efektów -- są zupełnie inne. Niewątpliwie model oparty na pozycji generuje sztywniejszą tkaninę niż jego rywal. Czasem skutkuje to omówionymi wyżej błędami, lecz nie występują tu mikrodrgania widoczne na prawym dolnym zrzucie ekranu rysunku \ref{pic_6_2}. Duże znaczenie ma także szybkość samej animacji tkaniny -- powinna ona opadać i~reagować na interakcje z~poruszającymi się obiektami tak szybko, jak w~rzeczywistości. Mimo swoich anomalii oraz trudności w uzyskaniu odpowiednio elastycznego modelu, dla gęstszych siatek metoda masy na sprężynie daje lepsze rezultaty wizualne. Z~drugiej strony, metodę opartą na pozycji cechuje dużo łatwiejsza regulacja elastyczności i~większa stabilność, jednak mogą się tu pojawić problemy z~ustaleniem odpowiedniej szybkości animacji. Założono stałość parametru \(\delta t \) przesyłanego do symulatora. W~obu metodach dobranie parametrów dla uzyskania pożądanego zachowania jest tym łatwiejsze, im mniej wierzchołków posiada siatka.
		
		W~przypadku małej liczby krawędzi można zaobserwować niedokładne wykrywanie kolizji pomiędzy tkaniną a~prostopadłościanem (rysunek \ref{pic_6_2}, oba górne zrzuty ekranu). Nie jest to jednak regułą, jako że problem pojawia się też dla gęstszych siatek (rysunek \ref{pic_6_3}, lewy dolny obrazek). Tutaj jednak winę ponosi także brak implementacji siły tarcia, co sprawia, że wierzchołki prześlizgują się po prostych ściankach obiektu, rozciągając tkaninę i~tworząc coraz większe otwory w miejscu przebicia. Dla sfery okalającej, ze względu na jej obły kształt, problemy przebicia nie występują. Wyjątkiem są szybko poruszające się obiekty, które mogą zwyczajnie przeskoczyć przez tkaninę, w~jednym kroku obliczeń znajdując się przed nią, a~w~następnym -- już za. Należałoby zastosować ciągłą metodę wykrywania kolizji, bardziej skomplikowaną matematycznie, lecz usuwającą takie zjawiska.
		
		\myownfigure{Wygląd tkaniny po przeniknięciu przez prostopadłościan (model oparty na pozycji).}{figures/pic_6_4.png}{0.38}{pic_6_4}
		
		Co gorsza, w~przypadku prostopadłościanu ześlizgiwanie się wierzchołków może stopniowo prowadzić także do kompletnego przeniknięcia tkaniny przez obiekt kolizyjny, poprzez powolne przenikanie przez niego ściąganych w dół kolejnych punktów masy. Efekt tego widać na rysunku \ref{pic_6_4}. Zdecydowanie sytuację poprawiłoby wprowadzenie siły tarcia bądź dokładniejszej metody detekcji kolizji, gdzie brane pod uwagę byłyby trójkąty siatki, a~nie tylko same wierzchołki.
		
		Zieloną ramką oznaczono fragment tkaniny, gdzie wystąpił błąd rozwiązywania kolizji wewnętrznych. Widać, że przeniknął on przez inną część modelu i~zawinął się w drugą stronę. Stało się tak z~powodu bardzo uproszczonej techniki sprawdzania tych kolizji, biorącej pod uwagę tylko cztery sąsiednie wierzchołki. Najprostszym rozwiązaniem problemu byłoby zwiększenie ich liczby, jednak wiąże się to z~większym kosztem obliczeniowym. Wyjściem jest także zastosowanie innej metody detekcji kolizji, o~której mowa w~poprzednim akapicie.
			
	%	\subsection{Model masy na sprężynie}
	%	\label{t:wyniki:efektwiz:ms}
			
			
	%	\subsection{Model oparty na pozycji}
	%	\label{t:wyniki:efektwiz:pb}

\chapter{Wyniki testów}
\label{t:wyniki}

	\section{Metoda badawcza}
	\label{t:wyniki:metoda}
	
	\section{Wydajność}
	\label{t:wyniki:wydajnosc}
	
	\section{Wygląd i występowanie artefaktów}
	\label{t:wyniki:artefakty}
	
	% Technika subtrakcji kolorów może generować artefakty po użyciu tekstur.

\chapter{Podsumowanie i wnioski}
\label{t:wnioski}

	\section{Porównanie implementowanych algorytmów}
	\label{t:wnioski:porownanie}
	
		\subsection{Wydajność}
		\label{t:wnioski:porownanie:wydajnosc}
		
		\subsection{Wygląd i występowanie artefaktów}
		\label{t:wnioski:porownanie:wyglad}
	
	\section{Zastosowanie i ograniczenia implementowanych algorytmów}
	\label{t:wnioski:zastosowanie}
	
	% brak próbkowania - dużo większa wydajność	
	% użycie buforów rozmytych o bardzo zredukowanym rozmiarze - większa wydajność
	% uproszczenie procesów fizycznych - większa wydajność
	% rozmycie buforów zamiast SAT może dodatkowo bardzo poprawić wydajność
	
	% konieczność generowania SAT i warstw, chociaż mniejszych
	% mniej zgodny z fizyką efekt, bardziej bazujący na podobieństwie wyglądu
	% liczba warstw jest ograniczona i ma wpływ na wydajność
	% konieczność przetwarzania większej ilości danych niz w StatVO
	
	Fakt rezygnacji z próbkowania eliminuje lub znacznie ogranicza większość wymienionych w rozdziale \ref{t:algorytm:wady-ssdo} problemów. Zamiast kilkunastu czy kilkudziesięciu próbek na piksel ma miejsce tylko osiem samplowań, dysponujemy bowiem dwoma SAT -- bufora głębi i wektorów normalnych oraz bufora koloru.
	
	Prowadzi on do rezygnacji z próbkowania na rzecz modelu statystycznego, a tym samym zastąpienia przebiegów rozmycia przebiegami generowania SAT dla poszczególnych warstw.
	
	\section{Obszar przyszłych badań}
	\label{t:wnioski:przyszle}

\begin{thebibliography}{999}

\bibitem{global-illum} \emph{Understanding Global Illumination} \url{ https://www.pluralsight.com/blog/film-games/understanding-global-illumination}, stan na dzień 12.09.2017

\bibitem{ao} S. Zhukov, A. Iones, G. Kronin, \emph{An Ambient Light Illumination Model}, Eurographics Rendering Workshop, 1998

\bibitem{statvo} Quintjin Hendrickx, Leonardo Scandolo, Martin Eisemann, Elmar Eisemann, \emph{Adaptively Layered Statistical Volumetric Obscurance}, High Performance Graphics, 2015

\bibitem{ssdo} Tobias Ritschel, Thorsten Grosch, Hans-Peter Seidel, \emph{Approximating Dynamic Global Illumination in Image Space}, Association for Computing Machinery, Inc., 2009

\bibitem{sloan} B. Loos, J. Sloan, \emph{Volumetric obscurance}, ACM SIGGRAPH, 2010

\bibitem{sat} Marcos Slomp, Toru Tamaki, Kazufumi Kaneda, emph {Screen-Space Ambient Occlusion Through Summed-Area Tables}, First International Conference on Networking and Computing, 2010

\bibitem{crytek} M. Mittring, \emph{Finding next gen: Cryengine 2}, ACM SIGGRAPH, 2007

\bibitem{luna} Frank D. Luna, \emph{Projektowanie gier 3D. Wprowadzenie do technologii DirectX 11}, Helion, 2014

\bibitem{prefix-sum} Mark Harris, \emph{Parallel Prefix Sum (Scan) with CUDA}, Nvidia, 2007

\bibitem{directx0} \emph{Windows Graphics Pipeline} \url{https://msdn.microsoft.com/en-us/library/windows/desktop/ff476882(v=vs.85).aspx}, stan na dzień 12.09.2017

\bibitem{directx1} \emph{What's new in Direct3D 11} \url{https://msdn.microsoft.com/en-us/library/windows/desktop/ff476346(v=vs.85).aspx}, stan na dzień 12.09.2017

\bibitem{directcompute} Eric Young, \emph{DirectCompute Optimizations and Best Practices}, Nvidia, GPU Technology Conference, 2010

\end{thebibliography}


\chapter*{Abstract}

% what exactly is cloth simulation
% why was it created and where is it used
% usage of cloth simulation on mobile devices
% main purposes of this paper
% what is to be done
% every chapter's summary

The realistic simulation of cloths is nowadays a~key to produce good-quality, authentic graphical visualizations of various fabrics, such as characters' garment elements, flags or curtains. This can be computationally expensive, more and more as number of particles, which fabric is divided into, increases. The solution to this matter was to use GPU -- \emph{Graphic Processing Unit} and perform all calculations on this device. On PC platform, this technique proved to be much faster than the standard CPU approach.

The main purpose of this work is to check whether this solution could also be introduced on the mobile devices. Most of them nowadays also have their own specialized GPU chips, but will they prove to be computationally faster than mobile CPUs? Is it possible and worth one's while to create visually appealing and efficient cloth simulation here? And how big is the difference between PC and mobile platform in GPGPU performance? This paper answers these questions.

A~test application was created, one of its main purposes being the visualisation of two selected simulation methods -- mass-spring and position-based model. It was equally important to show cloth's collisions between other objects in scene and itself. The user is allowed to set various parameters that influence the simulation, such as the aforementioned method type, mesh density and dimensions or elasticity coefficient. He can also impact the movement of the cloth, swiping his finger along the device's touch screen, which is something unique to the mobile platform. To fully measure every important factor of the simulation, its three implementations were created -- one using GPU for computing and the other two using GPU, in sequential and multi-threaded approach. To have a~comparison between mobile and PC platform, a~PC version of the application was created, both similar and sharing as much code with each other as possible.

In chapter 2 of this paper there are described theoretical basis of incorporated simulation methods and collision resolving. General Purpose GPU Computing is also mentioned, along with GPU framework and a~comparison between it and a~CPU is made, in the matter of architecture and performance. Chapter 3 analyses abilities of mobile devices, also mentioning their unique UI capabilities after comparing test device -- LG Nexus 4 -- to an~example PC. All used APIs, libraries and most important functions are also described. Among them are OpenGL ES 3.0, OpenGL 3.3 and Android NDK, as test application is completely written in C++. Chapter 4 shows architecture of~its engine, magnifying characteristics of~the most important components. Algorithms of~the engine's operation are also given, including such actions as~updating scene's entities, communication with Android OS, rendering or UI. In Chapter 5 attention is turned to the cloth simulator only, describing its general work flow, divided into three stages -- particle movement computation, collision solving and recalculation of normal vectors. Every stage is then thoroughly described, with their most important fragments's code given. Finally, in chapters 6 and 7, results of test application's work are shown, in the matter of both simulation models' computation time, stability and visual appeal, considering all three implementations and two tested platforms. In the end it is proved that the cloth simulation can be implemented on mobile devices and the average one's GPU can perform very well, producing smooth animation of fabric's dense mesh, but not without a few important limitations. These include less useful API functions and shorter work time on battery as a result of intensive computations and tendency to overheating.

\lstlistoflistings

\listoffigures

\listoftables

\listofalgorithms

\newpage
\thispagestyle{empty}
\begin{textblock}{1}(-2.65,-1.65)
\includegraphics{figures/zal1.pdf}
\end{textblock}
\newpage
\thispagestyle{empty}
\begin{textblock}{1}(-2.65,-1.65)
	\includegraphics{figures/zal2.pdf}
\end{textblock}
\newpage
\thispagestyle{empty}
\begin{textblock}{1}(-2.65,-1.65)
	\includegraphics{figures/zal3.pdf}
\end{textblock}

\end{document}
