\chapter*{Abstract}

Developers of video games and simulations from the day one have been trying to improve visuals of their products. A few ideas arose to use global illumination for most realistic simulation of light distribution. Unfortunately, it is still computationally too expensive, even on modern hardware. One of the ways to simulate it without a lot of processing is called Screen-Space Ambient Occlusion. Many implementations of this technique were created, though few take into account direction and colour of the incoming light. One of the exceptions is a method called SSDO -- Screen-Space Directional Occlusion. Unfortunately, it suffers from the same drawbacks as its less realistic cousins, such as noise and banding while also remaining moderately expensive for computation. The main purpose of this work is to optimize basic SSDO method using technique called Statistical Volumetric Obscurance, enhancing its performance while retaining plausible visual effect.

A showcase application was created with two novel implementations of the SSDO algorithm. A basic SSDO method was also included for good measure. Program renders a moderately complex scene, with a lot of edges and cavities. User is able to control the camera, cycle through the techniques, see and judge them using displayed framerate counter.

In chapter \ref{t:teoria} theoretical informations were presented about SSAO, SSDO and Statistical VO techniques. Chapter \ref{t:algorytm} describes ideas behind the improvement of SSDO and usage of the statistical model. It also shows differences between two new techniques. In chapter \ref{t:technologie} all used technologies were described with a reasoning behind using them given. Chapter \ref{t:budowa} displays a basic architecture of the showcase application with a depiction of its main modules, especially rendering component. In chapter \ref{t:impl} every SSDO implementation was thoroughly explained with HLSL code listings supplied for most crucial elements. Chapter \ref{t:wyniki} shows the results of tests, giving information about the performance of every technique and presenting screenshots of their operation. In the last chapter \ref{t:wnioski} SSDO methods are compared and their work's results are described. Their limitations, chances for applications in practice and methods of future improvements are discussed.