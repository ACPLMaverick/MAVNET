\chapter{Wprowadzenie}
\label{t:wprowadzenie}

	% twórcy dążą do generowania realistycznej grafiki i wprowadzają oświetlenie globalne etc.
	% wymagający, niedokładny i podatny na błędy proces
	% ulepszenie wydajności kierunkowej okluzji
	
	Twórcy gier wideo i symulacji komputerowych od początku istnienia tych dziedzin dążyli do uzyskania obrazu jak najbardziej odpowiadającego rzeczywistości. Algorytmy obliczania oświetlenia wirtualnej sceny są prawdopodobnie najistotniejszymi elementami w~tym procesie. Ponieważ standardowy, oparty na śledzeniu promieni model światła globalnego był i ciągle jest zbyt kosztowny w aplikacjach czasu rzeczywistego, wielu twórców prześciga się w kreowaniu nowych metod przybliżania tej techniki. 
	
	Jedna z najpopularniejszych to SSAO -- okluzja otoczenia obliczana w przestrzeni ekranu, symulująca zacienienia w miejscach, do których w rzeczywistości docierałoby mniej światła. Powstało bardzo dużo algorytmów generujących SSAO, jednak dużą część z nich trapią problemy związane ze znaczącym wpływem na wydajność, artefaktami oraz szumami. Okluzja otoczenia w klasycznym ujęciu jest bowiem efektem o niskiej częstotliwości. Warto także wspomnieć, że nie wiąże się z w żaden sposób z procesem obliczenia oświetlenia, pomimo, iż generuje wiarygodny i przyjemny dla oka efekt.
	
	Jako ulepszenie omawianej techniki wprowadzono \emph{kierunkową} okluzję otoczenia (SSDO) \cite{ssdo}, której główną cechą jest połączenie z parametrami światła, takimi jak kierunek i kolor. W ramach tej metody obliczone zostaje także jedno odbicie światła pośredniego. Trapią ją jednak podobne bolączki, co SSAO. Oprócz tego stanowi dość kosztowny obliczeniowo proces. Głównym zadaniem niniejszej pracy jest poprawa wydajności SSDO przy jednoczesnym zachowaniu, lub nieznacznym pogorszeniu, walorów wizualnych. Dokonano tego przy pomocy statystycznego modelu wolumetrycznego przesłonięcia \cite{statvo} zamiast podejścia klasycznego.
		
	\section{Cel pracy}
	\label{t:wprowadzenie:cel}
	
	% wzięcie modelu SSDO
	% wzięcie StatVO i zaimplementowanie za jego pomocą	SSDO
	% finalnie dwie techniki różniące się procesem przygotowania danych
	% zbadanie wydajności wpływu wizualnego oraz porównanie
	% omówienie zastosowań, ograniczeń i możliwości rozwoju
	
	Istnieje wiele technik poprawy wydajności oraz ulepszenia efektów generowanych przez okluzję otoczenia. Jedną z nich jest wspomniana wcześniej metoda statystyczna (StatVO). Niniejsza praca stawia sobie za zadanie polepszenie wydajności SSDO przy wykorzystaniu algorytmów StatVO. Główną zmianą będzie rezygnacja z próbkowania okolicy przetwarzanego piksela na rzecz wcześniejszego obliczania wartości średniej wokół niego. Dzięki temu nowa technika powinna być wolna od szumów i pracować dużo szybciej od oryginału. Algorytmów obliczania oświetlenia SSDO nie da się bezpośrednio zastosować w metodzie statystycznej, zaproponowano więc nowe wzory, za pomocą których osiągnięto zbliżony efekt wizualny.
	
	Opracowano i zaimplementowano dwie techniki statystycznej kierunkowej okluzji otoczenia. Różnią się one procesem przygotowania danych, a wspomniane wyżej wzory pozostają takie same. Zaimplementowano także bazową metodę SSDO dla celów porównawczych. Zbadano wydajność każdej techniki w postaci osiąganej liczby klatek na sekundę. Wartości te porównano ze sobą, omówiono przyczyny, dla których jedna z metod pracuje szybciej, a~druga dużo wolniej. 
	
	Przedyskutowano przewagi obu nowych technik nad pierwowzorem oraz ich wady. Zaproponowano rozwiązania najbardziej kluczowych problemów. Wzięto pod uwagę także efekt wizualny i zbadano różnice pomiędzy efektem oraz intensywnością zacienienia, wpływ kierunku i koloru światła na te czynniki oraz występujące artefakty. Na koniec omówione zostały zastosowania technik, ich ograniczenia oraz możliwości rozwoju.
	
	
	\section{Założenia}
	\label{t:wprowadzenie:zalozenia}
	
	% krótki opis aplikacji
	% krótki opis platformy i narzędzi
	
	Na potrzeby pracy stworzono aplikację prezentacyjną pokazującą działanie wszystkich trzech zaimplementowanych technik kierunkowej okluzji otoczenia. Renderuje ona trójwymiarową scenę, której geometria zawiera dużą liczbę zagięć, kantów oraz wnęk, za sprawą czego uzyskany zostaje bogaty efekt okluzji. Obiekty oświetlone są światłem kierunkowym obracającym się wokół ustalonej osi. Dzięki temu można obserwować zmiany i~przejścia SSDO. Użytkownik kontroluje położenie oraz obrót wirtualnej kamery przy użyciu klawiatury. Może także przełączać wyświetlanie zaimplementowanych trzech technik SSDO lub wyłączyć je. Aplikacja w formie tekstowej oraz liczbowej prezentuje liczbę klatek na sekundę oraz czas rysowania jednej ramki, co pozwala szybko i łatwo ocenić wydajność aktualnie oglądanej metody SSDO.
	
	Stworzona aplikacja to mały silnik renderujący z możliwością definiowania obiektów w~scenie, wczytywania oraz renderowania zasobów, takich jak siatki wielokątowe lub napisy. Na gotową ramkę obrazu nałożona może zostać dowolna liczba post-procesów. Da się także zmieniać parametry kamery. Moduł graficzny oświetla obiekty korzystając z techniki renderingu odroczonego.
	
	Aplikacja została stworzona na platformę Windows przy użyciu technologii Direct3D 11, WINAPI oraz kilku pomniejszych bibliotek. Ważną kwestią jest użycie właśnie tej wersji API graficznego Microsoftu, gdyż udostępnia on shadery obliczeniowe, kluczowe w implementacji statystycznej okluzji otoczenia.
	
	Urządzenie testowe to średniej klasy komputer PC.
	
	\section{Zakres pracy}
	\label{t:wprowadzenie:zakres}
	
	% streszczenia rozdziałów
	
	Aby ułatwić Czytelnikowi zrozumienie wszystkich poruszanych zagadnień, rozdział \ref{t:teoria}~zawiera w sobie niezbędną wiedzę teoretyczną. Krok po kroku zostanie omówiona istota działania okluzji otoczenia. Dalej wyjaśniona będzie koncepcja SSAO, czyli okluzji otoczenia w przestrzeni ekranu. Następnie naświetlona zostanie idea statystycznego wolumetrycznego przesłonięcia, wraz z kluczowymi wzorami. Bardzo istotnym elementem jest tu proces generowania tablic sum (SAT), którego definicja także będzie przytoczona. Na koniec dogłębnie wyjaśniono zasadę działania algorytmu kierunkowej okluzji otoczenia oraz podano wzory, na bazie których obliczone zostaje oświetlenie bezpośrednie oraz pośrednie.
	
	Rozdział \ref{t:algorytm} skupia się na głównej idei niniejszej pracy, czyli połączeniu techniki SSDO z ideą statystycznego wolumetrycznego przesłonięcia (StatVO). Najpierw omówione zostają wady SSDO, a następnie opisano, w jaki sposób StatVO zostało do tej metody zastosowane. Podano kluczowe wzory dla obu przebiegów oświetlenia, omówiono różnice pomiędzy dwoma zaimplementowanymi technikami oraz poruszono kwestię użycia przez nie pamięci.
	
	Rozdział \ref{t:technologie} to opis zastosowanych w prezentowanej aplikacji technologii. Dla każdej z~nich zamieszczono krótki opis, a następnie wyjaśniono, jaką rolę pełni ona w programie oraz dlaczego to właśnie jej użyto.
	
	W rozdziale \ref{t:budowa} zaprezentowano architekturę aplikacji. Podano główne komponenty programu oraz przedstawiono połączenia pomiędzy nimi przy pomocy uproszczonych diagramów UML. Każdy z kluczowych komponentów został dokładnie omówiony, przy czym szczególną uwagę poświęcono modułowi renderingu odroczonego oraz post-procesów.
	
	Rozdział \ref{t:impl} skupia w sobie omówienie implementacji trzech występujących w niniejszej pracy technik generowania SSDO. W każdym przypadku pokazano, jak zostały zrealizowane założenia projektowe. Podano algorytmy działania oraz listę wszystkich czynności, które są wykonywane, by osiągnąć założony efekt. Najbardziej kluczowe elementy implementacji, w szczególności realizacje omówionych w rozdziałach \ref{t:teoria} i \ref{t:algorytm} wzorów, uzupełniono listingami kodu HLSL.
	
	W rozdziale \ref{t:wyniki} zaprezentowano wyniki działania każdego algorytmu. Wydajność przedstawiono w tabeli i na wykresie, a za jednostkę miary przyjęto liczbę klatek rysowanych przez program w ciągu sekundy. Następnie za pomocą zrzutów ekranu pokazano wizualny efekt działania każdej z technik.
	
	Ostatni rozdział \ref{t:wnioski} kryje w sobie przede wszystkim podsumowanie efektów działania zaimplementowanych technik. Uzyskane wyniki zostają omówione i porównane ze sobą. Szczególny nacisk położono na wady oraz ograniczenia omawianych metod. Przedyskutowano także ich zastosowania w praktyce, a także nakreślono kierunek dalszych możliwych badań w omawianym temacie.