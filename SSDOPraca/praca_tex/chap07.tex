\chapter{Wyniki testów}
\label{t:wyniki}

	\section{Metoda badawcza}
	\label{t:wyniki:metoda}
	
	% Pomiar FPS 30 sekund dla tego samego ujęcia
	% Porównanie liczby FPS z liczbą FPS bez postprocesu pozwala ocenić wydajność
	% Screeny oceniane organoleptycznie, wzorem jest technika A.
	
	\section{Wydajność}
	\label{t:wyniki:wydajnosc}
	
	% C - 2130
	% B - 362
	% A - 793
	% NONE - 2880
	
	% Tabela
	
	% Wykres słupkowy
	
	\pgfplotstabletypeset
	[
	every head row/.append style={before row={%
			\caption{Wydajność poszczególnych technik SSDO.}%
			\label{tab_7_A}\\\toprule}}
	]
	{chart_7_A.dat}
	
	
	\section{Wygląd i występowanie artefaktów}
	\label{t:wyniki:artefakty}
	
	% Screeny dla każdej techniki: dwa rodzaje zacienienia (schody i okna), rozjaśnienia (boksy) oraz screen ukazujący błędy
	
	\pagebreak
	
	\myownfigure{Efekty działania techniki SSDO-A.}{figures/fig_7_A.png}{0.35}{fig_7_A}
	
	\pagebreak
	
	\myownfigure{Efekty działania techniki SSDO-B.}{figures/fig_7_B.png}{0.35}{fig_7_B}
	
	\pagebreak
	
	\myownfigure{Efekty działania techniki SSDO-C.}{figures/fig_7_C.png}{0.35}{fig_7_C}
	
	\pagebreak
	
	\myownfigure{Porównanie technik z obrazem bez okluzji.}{figures/fig_7_D.png}{0.35}{fig_7_D}
	
	\pagebreak