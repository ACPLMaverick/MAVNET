\chapter{Wyniki testów}
\label{t:wyniki}

	\section{Metoda badawcza}
	\label{t:wyniki:metoda}
	
	% Pomiar FPS 30 sekund dla tego samego ujęcia
	% Porównanie liczby FPS z liczbą FPS bez postprocesu pozwala ocenić wydajność
	% Screeny oceniane organoleptycznie, wzorem jest technika A.
	
	Przyjęta w niniejszej pracy metoda badawcza składa się z dwóch oddzielnych procesów. Jako pierwsza została zmierzona wydajność każdego z algorytmów (podrozdział \ref{t:wyniki:wydajnosc}). Uzyskane wartości pozwalają na ocenę przydatności danej techniki w zastosowaniach praktycznych, np. w silniku gry. Im szybciej działa metoda, tym więcej zaimplementować można podobnych jej technik wykonujących obliczenia podczas pracy programu i w ten sposób uzyskać bardziej przyjemny dla oka efekt graficzny.
	
	Druga część metody badawczej (podrozdział \ref{t:wyniki:artefakty}) polega na wykonaniu osobnych zrzutów ekranu w newralgicznych miejscach sceny aplikacji i wizualnej ocenie algorytmów. Uwagę zwrócono przede wszystkim na odstępstwa technik SSDO-B i SSDO-C od ,,oryginalnej'' techniki SSDO-A oraz na częstość i rozmiary występujących artefaktów graficznych. Na koniec porównano efekty działania wszystkich trzech metod z obrazem wygenerowanym bez użycia którejkolwiek z nich.
	
	Na koniec, w rozdziale \ref{t:wnioski} następuje porównanie wszystkich trzech technik pod względem obu omówionych czynników, dyskusja na temat ich zastosowania w praktyce oraz ograniczeń, a także określenie tego, co można poprawić w przyszłości.
	
	\section{Wydajność}
	\label{t:wyniki:wydajnosc}
	
	% C - 2130
	% B - 362
	% A - 793
	% NONE - 2880
	
	% Tabela
	
	% Wykres słupkowy
	
	Wydajność określono jako wielkość odwrotnie proporcjonalną do czasu wykonania danej techniki. Im szybciej pracuje algorytm, tym większa jego wydajność. Główna jednostka przyjęta podczas wykonywania pomiarów to liczba FPS (\emph{Frames Per Second}) określająca, ile ramek obrazu może zostać wygenerowanych w czasie jednej sekundy. Wartość ta jest wprost proporcjonalna do wydajności algorytmu. Drugorzędna jednostka zawarta w tabeli \ref{tab_7_A} to czas w milisekundach potrzebny na obliczenia i narysowanie sceny. Jest to więc wartość odwrotna do FPS.
	
	Metoda pomiarowa polega na obliczaniu w każdym kroku głównej pętli programu czasów, jakie potrzebne są na jej wykonanie. Dzieje się to od 10 do 40 sekundy pracy aplikacji, czyli trwa w sumie 30 sekund. Następnie przy użyciu wyliczonych wartości zostaje uzyskany średni czas przetwarzania jednego kroku programu w przyjętym interwale. Na koniec wartość ta jest przekształcana do FPS i zapisywana w tabeli. Dzieje się tak dla każdej omawianej techniki. Wykonano także pomiary wydajności programu bez włączonej jakiejkolwiek metody SSDO. Pozwala to uzyskać punkt odniesienia przy ich porównaniu i zobaczyć, jak bardzo po użyciu każdej spada liczba FPS.
	
	Podczas pomiarów przyjęto stałą rozdzielczość tylnego bufora równą 1280x720 pikseli. Kamera zawsze znajduje się w tym samym miejscu i rysuje te same obiekty. Aplikację uruchamiano na komputerze z procesorem Core i7-6700K 4 GHz, 16 GB RAM i kartą graficzną GeForce GTX 970 z 4 GB pamięci. Ponieważ wszystkie algorytmy pracują w~przestrzeni ekranu, liczba renderowanych trójkątów nie ma wpływu na ich wydajność.
	
	W tabeli \ref{tab_7_A} oraz na wykresie \ref{wykr_7_A} przedstawiono uzyskane wyniki pomiarów. Poprzez \(ms_{r}\) rozumie się liczbę milisekund potrzebnych na wszystkie obliczenia związane z daną techniką bez uwzględnienia pozostałej części aplikacji.
	
	% rozdzielczość, kamera, sprzęt.
	
	\pgfplotstabletypeset
	[
	columns/$$/.style={string type},
	every head row/.append style={before row={%
			\caption{Wydajność poszczególnych technik SSDO.}%
			\label{tab_7_A}\\\toprule}}
	]
	{chart_7_A.dat}
	\($$\)
	
	\pgfplotstableread
	{
		method fps
		SSDO-A 793
		SSDO-B 362
		SSDO-C 2130
		Brak 2880
	}\loadedtable

	\pgfplotstableread[header=false]{
		793 SSDO-A
		362 SSDO-B
		2130 SSDO-C
		2880 Brak
	}\datatable
	
	\begin{figure}[H]
		\begin{tikzpicture}
		\begin{axis}[
			xbar, bar shift=0pt,
			enlarge y limits=0.2,
			xmin=0,
			xtick={0, 400, 800, 1200, 1600, 2000, 2400, 2800, 3200, 3600},
			ytick={0,...,4},
			yticklabels from table={\datatable}{1},
			xmajorgrids = true,
			bar width=8mm, 
			width=14cm, height=6.5cm, 
			xlabel={FPS},
			ylabel={Technika SSDO},
			nodes near coords, nodes near coords align={horizontal},
			]
			
			\pgfplotsinvokeforeach{0,...,4}{
				\addplot table [select row=#1, y expr=#1] {\datatable};
			}
		\end{axis}
		\end{tikzpicture}
		\caption{Wykres przedstawiający liczbę osiąganych FPS przez każdy z algorytmów. Źródło: opracowanie własne.}
		\label{wykr_7_A}
	\end{figure}

	\pagebreak
	
	\section{Wygląd i występowanie artefaktów}
	\label{t:wyniki:artefakty}
	
	% Screeny dla każdej techniki: dwa rodzaje zacienienia (schody i okna), rozjaśnienia (boksy) oraz screen ukazujący błędy
	\raggedbottom
	Kryterium wyglądu jest subiektywne, jednak wyróżnić można kilka cech wygenerowanych obrazów, które da się w pewnym stopniu ocenić, a nawet porównać ze sobą. Wśród nich należy wymienić skłonność do występowania artefaktów, czyli zacienień w miejscach, gdzie zgodnie z zasadą rozchodzenia się światła nie powinno ich być. Kolejnymi czynnikami są: stopień zaszumienia okluzji, gładkość jej gradientu podczas przechodzenia z ciemnych do jasnych miejsc oraz stopień zacienienia przy tych samych parametrach wejściowych. Ocenie podlega także kształt i siła generowanego oświetlenia pośredniego.
	
	W celu ukazania efektu pracy poszczególnych technik oraz porównania osiągniętych przez nich efektów wizualnych, dla każdego algorytmu sporządzono cztery zrzuty ekranu, oznaczone kolejno literami A, B, C i D. Trzy z nich wykonane zostały zawsze w tym samym położeniu kamery oraz światła. Dwa pierwsze służą ocenie wyglądu okluzji kierunkowej, trzeci -- oświetlenia pośredniego. Czwarty zrzut ekranu jest różny dla każdej z technik a jego zadanie to prezentacja błędów, jakie mogą powstać podczas renderingu. Kolorowe okręgi oraz prostokąty naniesiono na przedstawione obrazy w programie graficznym. Umożliwia to łatwe odniesienie do interesujących elementów w~rozdziale \ref{t:wnioski:porownanie:wyglad}.
	
	Na koniec przedstawiono cztery zrzuty ekranu z tego samego miejsca w scenie. Trzy pierwsze odpowiadają kolejnym technikom, czwarty został wyrenderowany bez użycia żadnej. Służy to porównaniu siły okluzji i uwidocznieniu, jak algorytmy wpływają na finalny wygląd sceny oraz podkreślenie kształtów geometrii.

	\pagebreak	\flushbottom
	
	\myownfigure{Efekty działania techniki SSDO-A.}{figures/fig_7_A.png}{0.35}{fig_7_A}
	
	\pagebreak
	
	\myownfigure{Efekty działania techniki SSDO-B.}{figures/fig_7_B.png}{0.35}{fig_7_B}
	
	\pagebreak
	
	\myownfigure{Efekty działania techniki SSDO-C.}{figures/fig_7_C.png}{0.35}{fig_7_C}
	
	\pagebreak
	
	\myownfigure{Porównanie technik z obrazem bez okluzji.}{figures/fig_7_D.png}{0.35}{fig_7_D}
	
	\pagebreak